%
%  5章 システムコール
%
\chapter{システムコール}

\tacos のシステムコールを呼び出す関数です.
{\bf トランスレータ版では使用できません.}
\|#include <syslib.hmm>|と書いた後で使用します.

\section{プロセス関連}

\tacos では,
\|exec|で新しいプロセスを作ると同時に新しいプログラムを実行します.
UNIXの\|fork-exec|方式とは異なります.

%\tac にはベースレジスタや多重仮想記憶のような機構がないので,
%\|fork|システムコールが実現できませんでした.

\subsection{exec}

\|path|でプログラムファイルを指定し,
新しいプロセスで新しいプログラムの実行を開始します.
\|argv|は,開始するプログラムの\|main|関数の
第2引数(\|char[][]argv|)に渡される文字列配列です.
\|envp|は,開始するプログラムに渡す環境変数です.
\|argv|と同じ要領で文字列配列を渡します.
\|exec|は正常なら\|0|,エラー発生なら負のエラー番号を返します.

\begin{quote}
\begin{verbatim}
#include <syslib.hmm>
public int exec(char[] path, void[] argv, void[] envp);
\end{verbatim}
\end{quote}

下に使用例を示します.
\|argv[0]|にプログラム名,
\|argv[1]|に第1引数のように格納します.
最後に\|null|を格納します.
環境変数に興味がない場合は,
自身の環境変数(\|environ|)を子プロセスに渡します.

\begin{mylist}
\begin{verbatim}
  #include <stdlib.hmm>
  #include <syslib.hmm>
  char[][] args = {"prog", "param1", "param2", null};
  public int main() {
    exec("/bin/prog.exe", args, environ);
    return 1;
  }
\end{verbatim}
\end{mylist}

子プロセス側のプログラム(prog.cmm)は次のようになります.

\begin{mylist}
\begin{verbatim}
public int main(int argc, char[][]argv, char[][]envp) {
  int c = argc;       // 前のプログラムで起動されたとき 3
  char[] s = argv[1]; // 前のプログラムで起動されたとき "param1"
  return 0;
}
\end{verbatim}
\end{mylist}

\subsection{\ul exit}

\|_exit|はプログラム(プロセス)を終了します.
\|_exit|は入出力のバッファをフラッシュしません.
\|_exit|は緊急終了用に使用し,
普通は標準ライブラリの\|exit|を使用します.

\|status|は,親プロセスに返す終了コードです.
\|0|が正常終了の意味,\|1|以上はユーザが決めた終了コード,
負の値は\tabref{chap4:err}に示す記号名で定義されています.
負の値を返すと親プロセスがシェルの場合,
シェル側でエラーメッセージを表示してくれます.

\begin{quote}
\begin{verbatim}
#include <syslib.hmm>
public void _exit(int status);
\end{verbatim}
\end{quote}

\subsection{wait}

\|wait|は子プロセスの終了を待ちます.
\|stat|には大きさ1の\|int|配列を渡します.
子プロセスが終了した際,\|stat[0]|に終了コードが書き込まれます.
\|wait|は正常なら\|0|,エラー発生なら負のエラー番号を返します.

\begin{quote}
\begin{verbatim}
#include <syslib.hmm>
public int wait(int[] stat);
\end{verbatim}
\end{quote}

\subsection{sleep}

\|sleep|はプロセスを指定された時間だけ停止します.
\|ms|はミリ秒単位で停止時間を指定します.
\|ms|に負の値を指定すると\|EINVAL|エラーになります.
それ以外では,\|sleep|は\|0|を返します.

\begin{quote}
\begin{verbatim}
#include <syslib.hmm>
public int sleep(int ms);
\end{verbatim}
\end{quote}

\section{ファイル操作}

\tacos は,マイクロSDカードのFAT16ファイルシステムを扱うことができます.
VFATには対応していません.

\subsection{creat}

\|creat|は新規ファイルを作成します.
\|path|は新しいファイルのパスです.
\|creat|は正常なら\|0|,エラー発生なら負のエラー番号を返します.

\begin{quote}
\begin{verbatim}
#include <syslib.hmm>
public int creat(char[] path);
\end{verbatim}
\end{quote}

\subsection{remove}

\|remove|はファイルを削除します.
\|path|は削除するファイルのパスです.
\|remove|は正常なら\|0|,エラー発生なら負のエラー番号を返します.

\begin{quote}
\begin{verbatim}
#include <syslib.hmm>
public int remove(char[] path);
\end{verbatim}
\end{quote}

\subsection{mkDir}

\|mkDir|は新規のディレクトリを作成します.
\|path|は新しいディレクトリのパスです.
\|mkDir|は正常なら\|0|,エラー発生なら負のエラー番号を返します.

\begin{quote}
\begin{verbatim}
#include <syslib.hmm>
public int mkDir(char[] path);
\end{verbatim}
\end{quote}

\subsection{rmDir}

\|rmDir|はディレクトリを削除します.
\|path|は削除するディレクトリのパスです.
\|rmDir|は正常なら\|0|,エラー発生なら負のエラー番号を返します.
削除するディレクトリが空でない場合はエラーになります.

\begin{quote}
\begin{verbatim}
#include <syslib.hmm>
public int rmDir(char[] path);
\end{verbatim}
\end{quote}

\subsection{stat}

\|stat|はファイルのメタデータを\|Stat|構造体に読み出します.
エラー発生なら負のエラー番号を返します.

\begin{quote}
\begin{verbatim}
#include <syslib.hmm>
public int stat(char[] path, Stat stat);
\end{verbatim}
\end{quote}

\|Stat|構造体は\|syslib.hmm|から間接的にインクルードされる
\|sys/fs.hmm|ファイル中で以下のように宣言されます.
\|attr|はFAT16ファイルシステムの
ディレクトリエントリから読みだしたファイルの属性です.
詳しい意味はFAT16ファイルシステムの文献を参照してください.

\begin{quote}
\begin{verbatim}
struct Stat {   // FAT16ファイルシステムからファイルの情報を取り出す.
  int attr;     // read-only(0x01),hidden(0x02),directory(0x10) 他
  int clst;     // ファイルの開始クラスタ番号
  int lenH;     // ファイル長上位16ビット
  int lenL;     // ファイル長下位16ビット
};
\end{verbatim}
\end{quote}

\section{ファイルの読み書き}

ファイルの読み書きには,
通常は\pageref{chap4:stdio}ページの標準入出力ライブラリ関数を用います.
以下のシステムコールは,主にライブラリ関数の内部で使用されます.

\subsection{open}

\|open|はファイルを開きます
\|path|は開くファイルのパスです.
\|mode|には\|O_RDONLY|,\|O_WRONLY|,\|O_APPEND|のいずれかを指定します.
\|open|は正常なら非負のファイル記述子,
エラー発生なら負のエラー番号を返します.
ファイルが存在しない場合は,どのモードでもエラーになります.
新規ファイルに書き込みたい場合は,
事前に\|creat|システムコールを用いてファイルを作成しておく必要があります.

\|open|はディレクトリを\|O_RDONLY|モードで開くことができます.
ディレクトリは\pageref{chap4:readDir}ページの\|readDir|関数で読みます.

\begin{quote}
\begin{verbatim}
#include <syslib.hmm>
public int open(char[] path, int mode);
\end{verbatim}
\end{quote}

\subsection{close}

\|close|は\|open|で開いたファイルを閉じます.
\|fd|は閉じるファイルのファイル記述子です.
\|close|は正常なら\|0|,エラー発生なら負のエラー番号を返します.

\begin{quote}
\begin{verbatim}
#include <syslib.hmm>
public int close(int fd);
\end{verbatim}
\end{quote}

\subsection{read}

\|read|は\|open|を用い\|O_RDONLY|モードで開いたファイルから
データを読みます.
\|fd|はファイル記述子です.
\|buf|はデータを読み込むバッファ,
\|len|はバッファサイズ(バイト単位)です.
\|read|は正常なら読み込んだバイト数,
エラー発生なら負のエラー番号を返します.
EOFでは\|0|を返します.

\begin{quote}
\begin{verbatim}
#include <syslib.hmm>
public int read(int fd, void[] buf, int len);
\end{verbatim}
\end{quote}

\subsection{write}

\|write|は\|open|を用い\|O_WRONLY|,
\|O_APPEND|モードで開いたファイルへデータを書き込みます.
\|fd|はファイル記述子です.
\|buf|は書き込むデータが置いてあるバッファ,
\|len|は書き込むデータのサイズ(バイト単位)です.
\|write|は正常なら書き込んだバイト数,
エラー発生なら負のエラー番号を返します.

\begin{quote}
\begin{verbatim}
#include <syslib.hmm>
public int write(int fd, void[] buf, int len);
\end{verbatim}
\end{quote}

\subsection{seek}

\|seek|は\|open|を用い開いたファイルの読み書き位置を変更します.
\|fd|はファイル記述子です.
seek位置は,上位16bit(\|ptrh|)と
下位16bit(\|ptrl|)を組み合わせて指定します.
\|seek|は正常なら\|0|,
エラー発生なら負のエラー番号を返します.

\begin{quote}
\begin{verbatim}
#include <syslib.hmm>
public int seek(int fd, int ptrh, int ptrl);
\end{verbatim}
\end{quote}

\section{コンソール関連}

コンソール入出力には,
通常は\pageref{chap4:stdio}ページの標準入出力ライブラリ関数を用います.
以下のシステムコールは,主にライブラリ関数の内部で使用されます.
\|ttyRead|はライブラリ関数が\|stdin|からの読み込みをする場合に,
\|ttyWrite|はライブラリ関数が\|stdout|,\|stderr|への
書き込みをする場合にライブラリ関数内部で使用されます.

\subsection{ttyRead}

\|ttyRead|はキーボードから1行入力します.
読み込んだ内容は\|buf|で指定されるバッファに格納されます.
\|len|は\|buf|のバイト数です.
読み込んだ内容の最後に\|'\0'|は含まれませんが\|'\n'|は含まれます.
\|ttyRead|はキーボードから入力した文字数を返します.

\begin{quote}
\begin{verbatim}
#include <syslib.hmm>
public int ttyRead(void[] buf, int len);
\end{verbatim}
\end{quote}

\subsection{ttyWrite}

\|ttyWrite|は\|buf|の内容を画面に出力します.

\begin{quote}
\begin{verbatim}
#include <syslib.hmm>
public int ttyWrite(void[] buf, int len);
\end{verbatim}
\end{quote}

\subsection{ttyCtl}

\|ttyClt|はコンソールのモードを操作します.

\begin{quote}
\begin{verbatim}
#include <syslib.hmm>
public int ttyCtl(int cmd, int mode);
\end{verbatim}
\end{quote}

\|cmd|には\|TTYCTL_GETMODE|,\|TTYCTL_SETMODE|のどちらかを指定します.
\|TTYCTL_GETMODE|を指定した場合,\|ttyCtl|は現在のモードを返します.
\|TTYCTL_SETMODE|を指定した場合,\|ttyCtl|はモードを\|mode|に変更します.
\|ttyCtl|の返り値と\|mode|では,
モードを\|TTYCTL_MODE_COOKED|,\|TTYCTL_MODE_ECHO|,\|TTYCTL_MODE_NBLOCK|の
ビットマップで表現します.

\|TTYCTL_MODE_COOKED|は通常ONになっています.
このモードがONになっている場合,
コンソール入出力で\|'\r'|と\|'\n'|の間で適切な変換が行われます.
また,\|'\b'|(バックスペースキー)を用いた行編集ができます.

\|TTYCTL_MODE_ECHO|も通常ONになっています.
このモードがONになっている場合,
キーボードから入力した文字が画面にエコーバックされます.

\|TTYCTL_MODE_NBLOCK|は通常OFFになっています.
このモードがONになっている場合,\|ttyRead|が入力待ちになりません.

以下にプログラム例を示します.

\begin{quote}
\begin{verbatim}
int mode = ttyCtl(TTYCTL_GETMODE, 0);      // 現在のモードを取得
int noechoMode = mode & ~TTYCTL_MODE_ECHO;
ttyCtl(TTYCTL_SETMODE, noechoMode);       // NOECHOモードに変更
...
ttyCtl(TTYCTL_SETMODE, mode);             // 最初の状態に戻す
\end{verbatim}
\end{quote}

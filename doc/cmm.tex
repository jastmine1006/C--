%
% C-- 言語の解説書
%
% 2023.08.20 v.4.0.0 : バージョン番号を変更
% 2023.08.18 v.3.4.0 : char->intの自動型変換に関する記述
% 2021.03.15 v.3.3.0 : iMalloc, cMalloc, bMalloc, rMalloc など追記
%                      (?Mallocで作った領域はRTCで添字チェックする)
%                      トランスレータでローカル変数名がソースと同じになる
% 2020.12.11 v.3.2.9 : TTYCTL_MODE_NBLOCK を追加
%                      interruptの使用例のベクタ番号を実際に合わせる
% 2020.06.30 v.3.2.3 : カレントディレクトリ追加
%                      トランスレータ版の環境変数バグ訂正
% 2019.12.10 v.3.2.2 : 環境変数ライブラリ追加,exec()に環境変数の引数追加
%                      Statシステムコール追加
% 2019.10.18 v.3.2.1 : conRead,conWriteをttyRead,ttyWriteに変更,ttyCtlを追加
% 2019.03.18 v.3.2.0 : トランスレータ版に実行時エラーチェックを試験的に導入
% 2019.01.27 v.3.1.12: メモリ保護違反,I/O特権モード,in(),out(),htoi()等追記
% 2018.11.18 v.3.1.11: fsize()、printf(``%ld'')、32ビット演算を追記
% 2018.02.24         : fseek() の説明を追加
% 2018.01.27 v.3.1.10: ドライバの -nostdinc -I オプションを追記
% 2016.10.09 V.3.1.7 : cm2i の記述を追加、文法まとめを改良
%                    : Util-- のバージョン番号のルール変更
% 2016.09.18 V.3.1.6 : 中間言語の命令の名称を変更
%                    : vmLdLab => vmLdNam, vmLdStr => vmLdLab,
%                    : vmTmpLab => vmLab, vmName = > vmNam,
%                    : vmChar => vmChr
%                    : ニーモニックを変更 CHAR => CHR
% 2016.09.10         : 文字列クラス分類関数にトランスレータの場合を追記
% 2016.08.13 V.3.1.2 : ver.3.1.2 用に改訂
%                    : feofの仕様変更
% 2016.05.29 V.3.1.0 : ver.3.1 用に改訂
%                    : cm2c,cm2e,cm2v コマンドを追加しコンパイル例など変更
% 2016.03.12 V.3.0.0 : ver.3 用に改訂
%
% $Id$
%
\documentclass[a4paper,11pt,twocolumn]{ltjsbook}   % lualatex
\usepackage{graphicx}
\usepackage{amsmath}
%\usepackage{amssymb}
%\usepackage{bm}
%\usepackage{fancybox}
%\usepackage{fancyhdr}
%\usepackage{lastpage}
%\usepackage[usenames]{color}
%\usepackage{multicol}
%\usepackage{bigfoot}  % \footnote{} 中で \verb/../ が利用可能になる。
%\usepackage{listings}
%\usepackage{url}

%----------------------------------------------------------------------
% ハイパーリンク設定(印刷用では色使いに注意)
\usepackage[unicode=true,colorlinks=true,
     ,linkcolor=blue,urlcolor=blue]{hyperref}  % PDF用
%\usepackage[unicode=true,hidelinks]{hyperref}  % 印刷用

%----------------------------------------------------------------------
\def\|{\verb|}
\newcommand{\myfigure}[5]{
\begin{figure}[#1]
%\begin{figure}[tbp]
\begin{center}
\includegraphics[width=#2]{#3}
\vspace{-.3cm}
\caption{#4}
\label{fig:#5}
\end{center}
\end{figure}
}

%----------------------------------------------------------------------
\newcommand{\myfigureA}[5]{
\begin{figure*}[#1]
%\begin{figure*}[tbp]
\begin{center}
\includegraphics[width=#2]{#3}
\vspace{-.3cm}
\caption{#4}
\label{fig:#5}
\end{center}
\end{figure*}
}

%----------------------------------------------------------------------
\newcommand{\figref}[1]{図~\ref{fig:#1}}
\newcommand{\tabref}[1]{表~\ref{tab:#1}}
\newcommand{\ver}{Ver. 4.0.0}
\newcommand{\cmm}{{\tt C--}}
\newcommand{\cmml}{\cmm 言語}
\newcommand{\cmmc}{{\tt c--}}
\newcommand{\vcmmc}{{\tt vm-c--}}
\newcommand{\ccmmc}{{\tt c-c--}}
\newcommand{\rtccmmc}{{\tt rtc-c--}}
\newcommand{\icmmc}{{\tt ic-c--}}
\newcommand{\cme}{{\tt cm2e}}
\newcommand{\cmc}{{\tt cm2c}}
\newcommand{\cmv}{{\tt cm2v}}
\newcommand{\cmi}{{\tt cm2i}}
\newcommand{\cl}{{\tt C}言語}
\newcommand{\javal}{{\tt Java}言語}
\newcommand{\tac}{TaC}
\newcommand{\tacos}{TacOS}
\newcommand{\util}{{\tt Util--}}
\newcommand{\as}{{\tt as--}}
\newcommand{\ld}{{\tt ld--}}
\newcommand{\objbin}{{\tt objbin--}}
\newcommand{\objexe}{{\tt objexe--}}
\newcommand{\size}{{\tt size--}}
%\newcommand{\ul}{\underline{~~}}
\newcommand{\ul}{\_}
\newcommand{\lw}[1]{\smash{\lower2.0ex\hbox{#1}}}

%----------------------------------------------------------------------
\newenvironment{mylist}
{%\vspace{0.3cm}
\begin{center}\begin{tabular}{|c|}\hline\\\begin{minipage}{0.9\textwidth}}
{\end{minipage}\\\\\hline\end{tabular}\end{center}%\vspace{0.3cm}
}

%----------------------------------------------------------------------
\newenvironment{myminipage}
{\begin{minipage}{14.5cm}\vspace{0.2cm}\begin{center}}
{\end{center}\vspace{0.2cm}\end{minipage}}

%----------------------------------------------------------------------
\newenvironment{mytable}[3]
{\small\begin{table}[#1]\begin{center}\caption{#2}\vspace{0.2cm}\label{tab:#3}}
{\end{center}\end{table}}

%----------------------------------------------------------------------
\begin{document}
%\setlength{\textwidth}{14cm}
%\setlength{\textheight}{22cm}
\setlength{\oddsidemargin}{10pt}
\setlength{\evensidemargin}{-10pt}
%\setlength{\topmargin}{10pt}
%\setlength{\marginparsep}{15pt}
%\setlength{\parskip}{0.3cm}
\setlength{\headsep}{1cm}
\frontmatter

% 表紙
\title{プログラミング言語 \cmm \\\ver}
\author{徳山工業高等専門学校\\情報電子工学科}
\date{}

\maketitle

% 著作権表示
\thispagestyle{empty}
\onecolumn
~
\vfill
\begin{flushleft}
Copyright \copyright ~~ 2016 - 2023 by \\
Dept. of Computer Science and Electronic Engineering, \\
Tokuyama College of Technology, JAPAN
\end{flushleft}

\vspace{0.8cm}

本ドキュメントは*全くの無保証*で提供されるものである.上記著作権者および
関連機関・個人は本ドキュメントに関して,その適用可能性も含めて,いかなる保証
も行わない.また,本ドキュメントの利用により直接的または間接的に生じたいかな
る損害に関しても,その責任を負わない.
\setcounter{page}{0}

% 目次
\tableofcontents

% 本文
\mainmatter
\onecolumn
\include{chap1} % はじめに
% 
% 2章 C--言語開発環境のインストール
%
\chapter{\cmml 開発環境のインストール}

\cmml を体験するために,自分のパソコンで\cmm コンパイラを
使用できるようにしましょう.
{\bf 以下の順番で}作業し,開発環境をインストールしてください.


\section{ユーティリティのインストール}

{\tac}用のプログラムを作成するために,まず,
{\cmm}コンパイラ用ユーティリティをインストールする必要があります.
ソースコードは
\url{https://github.com/tctsigemura/Util--/}から入手します.

ダウンロードした配布物を展開し
「{\util}解説書」({\tt Util--/doc/umm.pdf})の手順に従いインストールします.
{\as},{\ld},{\objbin},{\objexe},{\size}の五つのプログラムが
{\tt /usr/local/bin}にインストールされます.

\section{コンパイラのインストール}
\label{chap2:compilerl}

{\cmm}コンパイラのソースコードは
\url{https://github.com/tctsigemura/C--/}から入手します.
ダウンロードした配布物を展開し以下の順にインストールします.

\subsection{コンパイラ本体のインストール}

%コンパイラ本体をインストールします.
{\tt C--/src}ディレクトリに移動し以下のように操作します.

\begin{mylist}
\begin{verbatim}
$ make
cc -std=c99 -Wall  -DDATE="\"`date`\"" -DVER=\"`cat ../VERSION`\"
...
$ sudo make install
Password:
install -d -m 755 /usr/local/bin
install -m 755 c-- /usr/local/bin
install -m 755 vm-c-- /usr/local/bin
...
\end{verbatim}
\end{mylist}

以上で,{\cme},{\cmc},{\cmi},{\cmv},{\cmmc},
{\ccmmc},{\rtccmmc},{\icmmc},{\vcmmc}の
九つのプログラムが作成され\|/usr/local/bin|にインストールされました.
{\cme},{\cmc},{\cmi},{\cmv}は,
{\cmml}をコンパイルするために使用するシェルスクリプトです.
これらのシェルスクリプトは,
与えられたファイルの拡張子から処理するべき手順を判断し,
コンパイラやユーティリティを自動的に実行します.

\begin{itemize}
\item {\cme}は,
{\cmm}プログラムを{\tac}で実行できる「{\tt .exe}」ファイルに変換します.
\item {\cmc}は,
{\cmm}プログラムをUNIXやmacOSで実行できるファイルに変換します.
\item {\cmi}は,
{\cmm}プログラムを中間コードに変換します.
\item {\cmv}は,
{\cmm}プログラムをスタックマシンのニーモニックに変換します.
\item {\cmmc}は,{\tac}用の{\cmml}コンパイラ本体です.
{\cmm}プログラムを{\tac}のアセンブリ言語に変換します.
通常{\cme}から呼び出され,ユーザが直接使用することはありません.
\item {\ccmmc}は,{\cmm}プログラムを{\cl}に変換するトランスレータです.
変換された{\cl}プログラムはUNIXやmacOSで実行できます.
通常{\cmc}から呼び出され,ユーザが直接使用することはありません.
\item {\rtccmmc}は{\ccmmc}と同様なトランスレータですが,
{\ccmmc}と異なり,
出力したプログラムに実行時エラーチェック機能を組み込みます.
実行時エラーチェックにより,
\|null|参照を使用した場合,
配列使用時に添字が範囲を超えていた場合に,
それを検出してプログラムが確実に停止します.
通常{\cmc}から呼び出され,ユーザが直接使用することはありません.
\item {\icmmc}は,コンパイラの勉強をしたい人のために,
中間言語(\pageref{app:vm}ページ)を出力して見せるコンパイラです.
通常{\cmi}から呼び出され,ユーザが直接使用することはありません.
\item {\vcmmc}は,コンパイラの勉強をしたい人のために,
中間言語(\pageref{app:vm}ページ)をよく反映した,
仮想スタックマシンのニーモニックを出力して見せるコンパイラです.
通常{\cmv}から呼び出され,ユーザが直接使用することはありません.
\end{itemize}

なお,これら九つのプログラムの使用方法は,
コマンドリファレンス(\pageref{app:command}ページ)で紹介します.

\subsection{ヘッダファイルのインストール}

{\cmml}プログラム用のヘッダファイルをインストールします.
\|C--/include|ディレクトリに移動し以下のように操作します.
\|/usr/local/cmmInclude|にインストールされます.

\begin{mylist}
\begin{verbatim}
$ sudo make install
Password:
install -d -m 755 /usr/local/cmmInclude
rm -f /usr/local/cmmInclude/*.hmm
install -m 644 *.hmm /usr/local/cmmInclude
...
\end{verbatim}
\end{mylist}

\subsection{ライブラリのインストール}

\|C--/lib|ディレクトリに移動し以下のように操作します.

\begin{mylist}
\begin{verbatim}
$ sudo make install
Password:
...
cm2e -c ../SrcTac/crt0.s
...
install -m 644 libtac.o /usr/local/cmmLib
...
install -m 644 *.[ch]  /usr/local/cmmLib/LibRtc
install -m 644 ../SrcC/cfunc.hmm /usr/local/cmmLib/LibRtc
install -m 644 ../SrcC/wrapper.c /usr/local/cmmLib/LibRtc
...
install -m 644 *.[ch] /usr/local/cmmLib/LibNortc
install -m 644 ../SrcC/cfunc.hmm  /usr/local/cmmLib/LibNortc
install -m 644 ../SrcC/wrapper.c  /usr/local/cmmLib/LibNotc
...
\end{verbatim}
\end{mylist}

\|/usr/local/cmmLib|ディレクトリには,以下のものがインストールされます.

\begin{itemize}
\item {\tt libtac.o}は{\cmml}で記述した{\tacos}用のライブラリ関数群です.
\item {\tt LibNortc}ディレクトリには,
トランスレータが実行時エラーチェックを行わない実行形式を作るときに
使用する{\cl}で記述したライブラリ関数などがインストールされます.
\item {\tt LibRtc}ディレクトリには,
トランスレータが実行時エラーチェックを行う実行形式を作るときに
使用する{\cl}で記述したライブラリ関数などがインストールされます.
\end{itemize}

\section{サンプルプログラムのコンパイル}

\|C--/samples/hello|ディレクトリに移動します.
{\tt hello.cmm}プログラムが準備してあります.
これをコンパイルして実行してみましょう.
{\tt hello.cmm}プログラムは以下のような{\cmml}プログラムです.

\begin{mylist}
\begin{verbatim}
//
// hello.cmm : C--のサンプルプログラム
//
#include <stdio.hmm>

public int main() {
  printf("hello,world\n");
  return 0;
}
\end{verbatim}
\end{mylist}

\subsection{コンパイルしてPCで実行}

\|C--/samples/hello|ディレクトリで以下のように操作すると,
{\tt hello.cmm}プログラムをコンパイルしてPCで実行することができます.
なお,{\cmc}は,「\ref{chap2:compilerl} コンパイラのインストール」で
インストールしたシェルスクリプトです.
コンパイラとユーティリティを自動的に起動して,
{\cmml}プログラムをPCで実行可能なプログラムに変換します.

\begin{mylist}
\begin{verbatim}
$ cm2c -o hello hello.cmm 
$ ./hello
hello,world
\end{verbatim}
\end{mylist}

\subsection{コンパイルしてTaCで実行}

\|C--/samples/hello|ディレクトリで以下のように操作すると,
{\tt hello.cmm}プログラムをコンパイルして{\tac}で実行可能な
{\tt hello.exe}を作ることができます.
なお,{\cme}は,「\ref{chap2:compilerl} コンパイラのインストール」で
インストールしたシェルスクリプトです.
コンパイラとユーティリティを自動的に起動して,
{\cmml}プログラムを{\tac}で実行可能なプログラムに変換します.
{\tt hello.exe}をメモリカードにコピーすると{\tac}で実行できます.

\begin{mylist}
\begin{verbatim}
$ cm2e -o hello hello.cmm 
$ ls -l
...
-rw-r--r--  1 sigemura  staff   138 May  5 01:03 hello.cmm
-rw-r--r--  1 sigemura  staff  8164 May 28 23:28 hello.exe
-rw-r--r--  1 sigemura  staff  5862 May 28 23:28 hello.map
\end{verbatim}
\end{mylist}

\subsection{いろいろな中間ファイル}

上の例で示した他に,{\tt hello.cmm}から数種類のファイルを作ることができます.

\begin{itemize}

\item {\tt hello.c}は,
{\cmc}を{\tt -S}オプション付きで実行して
{\tt hello.cmm}を変換した{\cl}ファイルです.
「\|$ cm2c -S hello.cmm|」のように実行します.
上記の{\tt hello}は,これを{\cl}コンパイラでコンパイルしたものです.

\begin{mylist}
\begin{verbatim}
#include <stdio.h>
#define _cmm_0S "hello,world\n"
int main(){
_printf(_cmm_0S);
return 0;
}
\end{verbatim}
\end{mylist}

\item {\tt hello.s}は,
{\cme}を{\tt -S}オプション付きで実行して
{\tt hello.cmm}を変換した{\tac}のアセンブリ言語プログラムです.
「\|$ cm2e -S hello.cmm|」のように実行します.
上記の{\tt hello.exe}は,
このファイルから{\cmm}コンパイラ用のユーティリティ({\tt Util--})を
用いて作成されました.

\begin{mylist}
\begin{verbatim}
_stdin  WS      1
_stdout WS      1
_stderr WS      1
.L1     STRING  "hello,world\n"
_main   PUSH    FP
        LD      FP,SP
        CALL    __stkChk
        LD      G0,#.L1
        PUSH    G0
        CALL    _printf
        ADD     SP,#2
        LD      G0,#0
        POP     FP
        RET
\end{verbatim}
\end{mylist}

\item {\tt hello.i}は,{\tt hello.cmm}を{\cmi}により
中間言語に変換したプログラムです.\\
「\|$ cm2i hello.cmm|」のように実行します.
中間言語の詳細は,
\pageref{app:vm}ページで紹介します.

\begin{mylist}
\begin{verbatim}
vmNam(25)
vmWs(1)
vmNam(26)
vmWs(1)
vmNam(27)
vmWs(1)
vmLab(1)
vmStr("hello,world\n")
vmNam(72)
vmEntry(0)
vmLdLab(1)
vmArg()
vmCallF(1, 69)
vmPop()
vmLdCns(0)
vmMReg()
vmRet()
\end{verbatim}
\end{mylist}

\item {\tt hello.v}は,{\tt hello.cmm}を{\cmv}により
仮想スタックマシンのニーモニックに変換したプログラムです.
中間言語とニーモニックは一対一に対応しています.\\
「\|$ cm2v hello.cmm|」のように実行します.
仮想スタックマシンのニーモニックの詳細は,
\pageref{app:vm}ページで紹介します.

\begin{mylist}
\begin{verbatim}
_stdin
        WS      1
_stdout
        WS      1
_stderr
        WS      1
.L1
        STRING  "hello,world\n"
_main
        ENTRY   0
        LDC     .L1
        ARG
        CALLF   1,_printf
        POP
        LDC     0
        MREG
        RET
\end{verbatim}
\end{mylist}
\end{itemize}

\subsection{Makefileファイル}

{\tt Makefile}は,
コンパイル手順を記述したファイルです.
数種類の例を示しますので参考にしてください.
なお,コマンド(\|em2e|と\|rm|)左の空白にはTABを用います.

\begin{mylist}
\begin{verbatim}
#
# Makefile.tac : C--言語からTacの実行形式に変換する手順
#

hello.exe: hello.cmm
        cm2e -o hello hello.cmm

clean:
        rm -f hello.exe hello.map
\end{verbatim}
\end{mylist}

\begin{mylist}
\begin{verbatim}
#
# Makefile.unix : C--言語からmacOSやUNIXの実行形式に変換する手順
#

hello: hello.cmm
        cm2c -o hello hello.cmm

clean:
        rm -f hello
\end{verbatim}
\end{mylist}

\begin{mylist}
\begin{verbatim}
#
# C-- 言語から a.out,.exe, .v ファイルを作る手順
#

all: hello hello.exe hello.v

# UNIX, macOS の a.out へ変換
hello: hello.cmm
        cm2c -o hello hello.cmm

# TacOS の実行形式を作る
hello.exe: hello.cmm
        cm2e -o hello.exe hello.cmm

# C--コンパイラの中間言語に変換する
hello.v: hello.cmm
        cm2v hello.cmm

clean:
        rm -f hello hello.c hello.s hello.v *.lst *.sym *.map \
	*.o *.exe *~

\end{verbatim}
\end{mylist}
 % インストール
% 
%  3章 C--言語の仕様
%
\chapter{\cmml の仕様}

システム記述言語として実績のある \cl を参考に\cmml は設計されました.
しかし,\cl は設計が古く分かりにくい仕様が多い言語でもあります.
そこで,設計が割と新しい \javal を参考に \cl の問題点を避けるようにしました.
最終的に \cmml は,
{\bf 「\javal の特徴を取り入れた簡易\cl」} になりました.

\section{コメント}

\cmml では,2種類のコメントを使用できます.
一つは \|/* 〜 */| 形式のコメント,
もう一つは \|// 〜 行末| 形式のコメントです.
以下に例を示します.
コメントは,空白を挿入できるところならどこにでも書くことができます.

\begin{mylist}
\begin{verbatim}
/*
 *  コメントの例を示すプログラム
 */
#include <stdio.hmm>             // printf を使用するために必要

public int main( /* 引数なし */ ) {
  printf("%d\n",10);
  return 0;                      // main は int 型なので
}
\end{verbatim}
\end{mylist}

\section{プリプロセッサ}

\cmml の仕様ではありませんが,
通常 \cmm コンパイラは \cl 用のプリプロセッサと組み合わせて使用します.
前の章で紹介した{\cme},{\cmc},{\cmi},{\cmv}は,
自動的にプリプロセッサを起動して{\cmm}プログラムを処理します.
プリプロセッサのお陰で,
\|#define|,\|#include|等のディレクティブを使用することができます.

\cmm コンパイラはプリプロセッサが処理した結果を受け取ります.
受け取った入力の,ヘッダファイル内部に相当する部分にエラーを見つけた場合でも,
正しくエラー場所をレポートできます.
また,\cl を出力するトランスレータとして動作する場合は,
出力した \cl プログラム中に適切な\|#include|ディレクティブを出力したり,
ヘッダファイル内部を省略して出力する等の処理をしています
\footnote{前章の hello.c の例を見てください.}
.

\section{データ型}

\cmml のデータ型は,\javal と同様に基本型と参照型に大別できます.
基本型には, int 型, char 型, boolean 型の3種類があります.
参照型には「配列型」,「構造体型」,「だまし型」があります.
参照型は \javal の参照型と良く似た型です.

\subsection{基本型}

int 型は整数値の表現に使用します.
char 型は文字の表現に使用します.
boolean 型は true と false のどちらかの値を取る論理型です.
\javal の boolean 型と同じものです.
違う型の間での代入はできません.
char 型から int 型だけ自動的な型変換を行いますが,
他の組合せでは自動的な型変換を行いません.

\subsubsection{int 型}
int型は16bit 符号付き2の補数表現2進数です.
(トランスレータ版では\cl の int 型に置き換えますのでCコンパイラに依存しますが,
{\cl}の int 型が32ビット,long 型が64ビットだと想定しています.
それ以外の環境ではテストしていません.)
-32768 から 32767 までの範囲の数値を表現することができます.
int 型変数は次のように宣言します.

\begin{mylist}
\begin{verbatim}
// int 型変数を宣言した例
int a;                       // int 型のグローバル変数
int b = 10;                  // 初期化もできる
public int main() {
  int c;                     // ローカル変数
  int d = 20;                // 初期化もできる
  return 0;
}
\end{verbatim}
\end{mylist}

int 型の定数は,
上のプログラム例にあるように 「\|10|」,「\|20|」と書きます.
また,16進数,8進数で書くこともできます.上の例と同じ値を,
16進数では「\|0xa|」,「\|0x14|」のように,
8進数では「\|012|」,「\|024|」のように書きます.
int 型は,四則演算,ビット毎の論理演算,シフト演算,比較演算等の
計算に使用できます.

\subsubsection{char 型}
char 型は 8bit の ASCII コードを格納するデータ型です.
\cmm の char 型は符号なしです.
char 型から int 型へは自動型変換ができます.
情報が失われる可能性がある int 型から char 型への型変換は
「\|chr|演算子」を用いて明示的に行う必要があります.
char 型に適用することができる計算は同値比較だけです.
他の演算は int 型へ自動型変換が行われた後に,
int 型の演算が実行されます.
「\|chr|演算子」,「\|ord|演算子」を用いた明示的な型変換により,
char 型の文字と int 型の ASCII コードの間で相互変換ができます
\footnote{以前のバージョンとの互換性を保つために
「\texttt{ord}演算子」が使用できます.}.

\begin{mylist}
\begin{verbatim}
// char 型変数を宣言した例
char a;                      // char 型のグローバル変数
char b = 'A';                // 初期化もできる
public int main() {
  char c;                    // ローカル変数
  char d = 'D';              // 初期化もできる
  int  i = ord(b);           // char 型から int 型への明示的な型変換
  i = 'a';                   // char 型から int 型への自動的な型変換
  c = chr(i);                // int 型から char 型へは明示的な型変換だけ可能
  c = chr(c + 0x20);         // c + 0x20 の結果は int 型
  return 0;
}
\end{verbatim}
\end{mylist}

char 型の定数は,
上のプログラム例にあるように 「\|'A'|」,「\|'D'|」と書きます.
制御文字を表現するために\tabref{chap3:escape}の
エスケープ文字が用意されています.

\begin{mytable}{tbhp}{エスケープ文字}{chap3:escape}
\begin{tabular}{ll}
\hline
\multicolumn{1}{c}{エスケープ文字} & \multicolumn{1}{c}{意味} \\
\hline
\|\n| & 改行 \\
\|\r| & 復帰 \\
\|\t| & 水平タブ \\
\|\x|16進数 & 文字コードを16進数で直接指定 \\
\|\|8進数 & 文字コードを3桁以内の8進数で直接指定 \\
\|\|文字 & 印刷可能な文字を指定(\|\'| や \|\\|等) \\
\hline
\end{tabular}
\end{mytable}

\subsubsection{boolean型}

\cl では int 型で論理値を表現しました.
そのため,条件式を書かなければならないところに間違って代入式を書いたミスを
発見できず苦労することがよくありました.
\cmml では論理型 boolean を導入したので,
このようなミスをコンパイラが発見できます.

\begin{mylist}
\begin{verbatim}
// C言語でよくある条件式と代入式の書き間違え
if (a=1) {
  ...
}
\end{verbatim}
\end{mylist}

boolean 型は論理演算と同値比較演算のオペランドになることができます.
boolean 型の定数値は,\|true|,\|false| と書き表します
\footnote{{\tt true}の内部表現は{\tt 1},{\tt false}の内部表現は{\tt 0}です.}
.
boolean 型も int 型と互換性がありません.
「\|bool|演算子」,「\|ord|演算子」に
よる明示的な型変換を用いると int 型との変換が可能です.

次に boolean 型の変数を宣言して使用する例を示します.

\begin{mylist}
\begin{verbatim}
// boolean 型の使用例
boolean b = true;     // true は定数

public int main() {
  boolean c = x==10;  // 比較演算の結果は boolean 型
  b = b && c;         // 論理演算の結果も boolean 型
  if (b) {            // 論理値なので条件として使用できる
    b = false;        // false も定数
    ...
  }
  int i = ord(b);     // boolean 型から内部表現への変換
  b = bool(i);        // 内部表現から boolean 型への変換
  ...
}
\end{verbatim}
\end{mylist}

\subsection{参照型}
\label{chap3:ref}

参照型は \cl のポインターに似た型です.
\javal の参照型と非常に良く似ています.
参照型には「配列型」,「構造体型」,「だまし型」があります.
参照型の値はインスタンスのアドレスです.
参照型の特別な値として何も指していない状態を表す \|null| があります.
\|null| の値はアドレスの \|0| です.
ここでは,配列,多次元配列,文字列, \|void| 配列,
構造体,だまし型について解説します.

\subsubsection{配列}
\label{chap3:array}
\cmml でも配列を使用することができます.
配列は「\|型名[]|」と宣言します.
例えば,int型やchar型の配列(参照変数)は次のように宣言します.

\begin{mylist}
\begin{verbatim}
// 配列の参照変数を宣言した例
int[] a;
char[] b;
\end{verbatim}
\end{mylist}

\myfigure{tbhp}{5cm}{array.pdf}{配列の構造}{chap3:array}

\figref{chap3:array}のように,
\cmml の配列は参照変数と配列インスタンスの組合せによって実現します.
上の宣言では,参照変数が生成されるだけです.
配列として使用するためには次の例のように
「\|array|」や「\|iMalloc|\footnote{
\texttt{iMalloc}はint型の配列を割り付けます.
char型配列を割り付ける\texttt{cMalloc},
boolean型配列を割り付ける\texttt{bMalloc},
参照型配列を割り付ける\texttt{rMalloc}も使用できます.
}」等を用いて
配列インスタンスを割り付ける必要があります.
「\|array|」は,配列インスタンスを静的に割り付けます.
「\|array|」は,関数の外で宣言される配列だけで使用できます.
「\|iMalloc|」は,配列インスタンスを実行時に動的にヒープ領域に割り付けます.
「\|iMalloc|」で割り付けた領域は,
使用後「\|free|」によって解放する必要があります.
%「\|free|」し忘れが予想されるので,
%配列は静的に割り付けることをおすすめします.

\begin{mylist}
\begin{verbatim}
// 配列インスタンスを割り付ける例
int[] a, b = array(10);      // 要素数10のint配列領域を割り付ける
public int main() {
  a = iMalloc(5);            // 要素数5のint配列領域を割り付ける
  ...
  free(a);                   // iMalloc した領域は忘れず解放する
  return 0;
}
\end{verbatim}
\end{mylist}

配列は次のようなプログラムでアクセスできます.
添字は 0 から始まります.

\begin{mylist}
\begin{verbatim}
// 配列をアクセスする例
b[9] = 1;
a[0] = b[9] + 1;
\end{verbatim}
\end{mylist}

トランスレータ版では,
実行時の添字範囲チェックがされます.
添字範囲エラーを検知した場合の実行例を次に示します.

\begin{mylist}
\begin{verbatim}
# トランスレータ版で配列の添字範囲エラーを検知した場合の実行例
$ cat err.cmm
#include <stdio.hmm>
int[] a = array(3);
public int main() {
  for (int i=0; i<=3; i=i+1)  // ループの実行回数が多すぎる
    a[i] = i;                 // ここでエラーが発生するはず
  return 0;
}
$ cm2c -o err err.cmm
$ ./err
err.cmm:5 Out of Bound idx=3
Abort trap: 6
\end{verbatim}
\end{mylist}

\subsubsection{多次元配列}

多次元配列は配列の配列として表現されます.
\figref{chap3:array2}に示すように,
1次元配列の参照を要素とした配列を使うと2次元配列になります.
下に,2次元配列を使用するプログラム例を示します.
プログラム中の \|a2| は,\|rMalloc|と\|iMalloc| によって,
\figref{chap3:array2}のようなデータ構造に作り上げられます.
「\|rMalloc|」と「\|iMalloc|」等を使用する場合はプログラムが複雑になってしまいますが,
長方形ではない配列も実現できます.
プログラム中 \|b2| は,「\|array|」を使用して
多次元配列に必要な配列インスタンスを割り付けた例です.
このように「\|array|」を使用すると,
多次元配列に必要な複雑なデータ構造を簡単に割り付けることができます.

%プログラムを簡単に記述するために,
%多次元配列は「\|array|」を使用して静的に割り付けることを
%強くおすすめします.

\myfigure{tbhp}{6cm}{array2.pdf}{2次元配列の構造}{chap3:array2}

\begin{mylist}
\begin{verbatim}
// 2次元配列の例
int[][] a2, b2 = array(4,3);       // 2次元配列の領域を割り付ける
public int main() {
  a2 = rMalloc(4);                 // 参照の配列を割り付ける
  for (int i=0; i<4; i=i+1) {
    a2[i] = iMalloc(3);            // int 型の1次元配列を割り付ける
  }
  ...
  for (int i=0; i<4; i=i+1) {
    free(a2[i]);                   // int 型の1次元配列を解放
  }
  free(a2);                        // 参照の配列を解放
  return 0;
}
\end{verbatim}
\end{mylist}

次のプログラムは多次元配列をアクセスする例です.
このプログラムは,\figref{chap3:array2}の右下の要素
(最後の要素)に 1 を代入したあと,
それを使用して \|b2| の要素の値を決めます.

\begin{mylist}
\begin{verbatim}
  a2[3][2] = 1;
  b2[3][2] = a2[3][2] + 1;
\end{verbatim}
\end{mylist}

\subsubsection{文字列}

\cl 同様に,文字列は char 型の配列として表現されます.
文字列は,文字コード \|0x00| の文字で終端されます.
次のプログラム例のように文字列定数を用いることができ,
また,文字列の終端を \|'\0'| との比較で判断できます.

\begin{mylist}
\begin{verbatim}
// 文字列の使用例
char[] str = "0123456789";
void putstr() {
  for (int i=0; str[i]!='\0'; i=i+1) {  // 文字列の終端まで
    putchar(str[i]);                    // 一文字づつ出力
  }
}
\end{verbatim}
\end{mylist}

\subsubsection{ void 配列}

\|void|型の配列は,どんな参照型とも互換性がある特別な型になります.
\cmml は型チェックが厳しいので,異なる型の間での代入はできません.
しかし,次のプログラム例のように \|void| 型配列を用いることにより,
異なる参照型の間で代入ができます.
また, \|malloc| 関数や \|free| 関数は,
\|void|型配列を使って実現されています.

\begin{mylist}
\begin{verbatim}
// 型変換の例
struct ABC { int a,b,c; };
struct XYZ { int x,y,z; };
ABC p = { 1, 2, 3 };

void putXYZ(XYZ r) {           // XYZ構造体を印刷する関数
  printf("(%d,%d,%d)\n",r.x,r.y,r.z);
}

public int main() {
  void[] tmp = p;              // tmpにABC参照型を代入可能
  XYZ q = tmp;                 // XYZ参照型にtmpを代入可能
  putXYZ(q);
  return 0;
}
\end{verbatim}
\end{mylist}

\subsubsection{構造体}

\label{chap3:struct}
\cmml の構造体は \javal のクラスからメソッドを取り除いたものに良く似ています.
構造体は次のプログラム例のように宣言します.
この例は双方向リストを構成するための \|Node| 型を宣言しています.
\cl の構造体宣言に良く似ていますが,
構造体名が型名になる点と,
構造体が参照型である点が異なります.
%\cl の構造体よりむしろメソッドを含まない \javal の class に似ていると言えます.

\begin{mylist}
\begin{verbatim}
struct Node {
  Nodo next;    // 自身と同じ型を参照
  Nodo prev;    // 自身と同じ型を参照
  int  val;     // ノードのデータ
};
\end{verbatim}
\end{mylist}

\|Node| 型の変数を宣言して使用する例を下に示します.
構造体名が型名になるので \|struct| を書かずに変数宣言します.
構造体メンバは,「\|参照.メンバ名|」形式で参照します.
\|new| の代わりに \|malloc| を使用する他は
\javal と良く似た書き方になります.

\begin{mylist}
\begin{verbatim}
public int main() {
  Node start;                         // 双方向リストのルート
  start = malloc(sizeof(Node));       // 番兵をリストに投げ込む
  start.next = start;                 // 番兵を初期化する
  start.prev = start;
  start.val  = 0;
  ...
}
\end{verbatim}
\end{mylist}

\subsubsection{だまし型}

内容不明のまま参照型を宣言できます.
他の言語で記述された関数を呼び出す場合など,
単に参照型として扱うだけで十分な場合があります.
\|typedef|の後に名前を書きます.
以下に使用例を示します.

\begin{mylist}
\begin{verbatim}
typedef FILE;              // FILE 型(だまし型)を宣言する

void f() {
  FILE fp = fopen(...);    // C言語の関数を呼び出す
  fputc('a', fp);          // C言語の関数を呼び出す
  ...
}
\end{verbatim}
\end{mylist}

\subsubsection{long 型の代用}

int の2倍のビット数の符号なし整数型を
int 型の大きさ2の配列で表現し long 型の代用とします.
{\tac}版では32ビット,トランスレータ版では64ビット符号なし整数です.
\|printf|,\|fsize|等の関数が使用します.

\begin{mylist}
\begin{verbatim}
int[] a = {12345, 6789};   // long 型の代用
\end{verbatim}
\end{mylist}



\section{関数}
\label{chap3:func}
\cmml でも,\cl 同様に関数を宣言して使用することができます.
\cl と比較して \cmml は厳格さを求めています.
つまり,先に宣言された関数しか呼び出すことができませんし,
引数の個数や型が完全に一致しないとコンパイル時エラーになります.
ライブラリ関数を使用する場合は,
呼び出す前に必ずプロトタイプ宣言をするか,
適切なヘッダファイルをインクルードする必要があります.
また,関数の型を省略することができません.
値を返さない関数は \|void|,
整数を返す関数は \|int|,
論理値を返す関数は \|boolean| 等と明示する必要があります.

\cl や \javal と同様に関数の仮引数は,関数の自動変数と同じように使用できます.
可変個引数の関数を宣言することもできます.
可変個引数関数の仮引数は「\|...|」と書きます.
可変個引数関数の内部では,
\pageref{chap4:args}ページで説明する
\|_args| 関数を使用して引数をアクセスします
\footnote{\cl へのトランスレータ版では,
可変個引数関数を定義することができません.}.
次にプログラム例を示します.

\begin{mylist}
\begin{verbatim}
#include <stdio.hmm>       // printf のプロトタイプ宣言が含まれる
int f() {                  // 引数の無い関数
  return 1;                // void 型以外の関数は必ずreturnが必要
}

void g(int x) {
  x = x * x;               // 仮引数は変数のように使用できる
  printf("%04x\n", x);     // プロトタイプ宣言が必要
}

public int main() {
  int x = f();             // 関数の呼び出し
  g(x);                    // 引数の型と個数が一致する必要がある
  return 0;                // void 型以外の関数は必ずreturnが必要
}
\end{verbatim}
\end{mylist}

\section{interrupt関数}
%特殊な種類の関数として interrupt 関数があります.
interrupt 関数は,OSカーネル等の割込みハンドラを\cmml
で記述するために用意しました.
コンパイラに\|-K|オプションを与えないと使用できません.

%普通の関数と異なり,
interrupt 関数はCPUのコンテキスト(フラグ,レジスタ等)を全く破壊しません.
関数の入口でコンテキストをスタックに保存し,出口で復旧します.
割込みにより起動される関数なので,
プログラムから呼び出すことはできません.
仮引数を宣言することもできません.
次の例のように関数型の代わりに \|interrupt| と書きます.
例中の \|main| 関数のように,
addrof 演算子(\pageref{chap3:addrof}ページ参照)や
\ul iToA 関数(\pageref{chap4:itoa}ページ参照)を使用して,
割込みベクタに interrupt 関数を登録します.

\begin{mylist}
\begin{verbatim}
interrupt timerHdr() {         // タイマー割込みハンドラのつもり
  ...
}
public int main() {
  int[] vect = _iToA(0xffe0);  // vect は割込みベクタの配列
  vect[1] = addrof(timerHdr);  // vect[1] はTimer1の割込みベクタ
  ...
}
\end{verbatim}
\end{mylist}

\section{変数}

\cmml の変数は静的な変数と自動変数の2種類です.
関数の外部で宣言した変数は全て静的な大域変数になります.
逆に,関数の内部で宣言した変数は
全てブロック内でローカルな自動変数になります.
関数の内部では,どこでも変数宣言が可能です.
ローカル変数の有効範囲はブロックの終わりまでです.
同じ名前の変数があった場合はローカル変数が優先されます.
ローカル変数同士の名前の重複は認めません(\javal と同じ規則,\cl と異なる).
次にプログラム例を示します.

\begin{mylist}
\begin{verbatim}
#include <stdio.hmm>
int   n = 10;                 // 静的な変数
int[] a = array(10);          // 静的な配列
public int main() {
  int i;                      // 関数内ローカル変数
  for (i=0;i<n;i=i+1) {       // 大域変数 n のこと
    int j = i * i;            // ブロック内ローカル変数
    printf("%d\n",j);
    int n = j * j;            // どこでも変数宣言可能
    printf("%d %d\n",j,n);    // ローカルな n のこと
  }                        
  printf("%d\n",n);           // 大域変数 n のこと
  return 0;
}
\end{verbatim}
\end{mylist}

\section{変数の初期化}

基本型の変数は,いつでも宣言と同時に初期化することができます.
参照型の変数は静的に割り付けられる場合だけ初期化できます.
構造体内部に入れ子になった参照型は\|null|で初期化することしかできません.
ただし名前表のようなデータ構造を作るために,
文字列での初期化だけは可能にしてあります.

\begin{mylist}
\begin{verbatim}
int     n = 10;                             // 基本型変数の初期化
int[]   a = { 1, 2, 3 };                    // 基本型配列の初期化
int[][] b = {{1,2,0},{1,2,3,0}};            // いびつな配列の初期化
struct List { int  val; List next; };       // 構造体の宣言
List    r = { 1, null };                    // 構造体変数の初期化
struct NameEntry { char[] name, int val; };
NameEntry[] weekTable = {                   // 名前表を作成する例
  {"Sun", 1}, {"Mon", 2}, {"Tue", 3}
};
public int main() {
  int i = 10;                               // 自動変数の初期化
  return 0;
}
\end{verbatim}
\end{mylist}

\section{public 修飾子}

関数,大域変数を他のコンパイル単位から参照できるようにします.
\|public|修飾子の付いていない関数や大域変数は
他のコンパイル単位からは見えないので,
重複を心配しないで自由に名前を付けることができます.
main関数はスタートアップルーチンから呼び出されるので,
必ず\|public|修飾をしなければなりません.

\begin{mylist}
\begin{verbatim}
int        n = 10;            // 同じ .cmm ファイル内だけで参照可
public int m = 20;            // 他の .cmm ファイルからも参照可

void f() { ... }              // 同じ .cmm ファイル内だけで参照可
public void g() { ... }       // 他の .cmm ファイルからも参照可

public void printf(char[] s, ...);  // ライブラリ関数は public

public int main() {          // main は必ず public
  f();
  g();
  printf("\n");
  return 0;
}
\end{verbatim}
\end{mylist}

\section{演算子}

\cmml には \cl をお手本に一通りの演算子が準備されています.
しかし,コンパイラを小さくする目的で,
レパートリーの多い代入演算子,
前置後置等の組合せが複雑なインクリメント演算子とデクリメント演算子を
省略しました.

\subsection{代入演算}

\cl や \javal には,たくさんの種類の代入演算子があり便利に使用できました.
\cmml では,コンパイラをコンパクトに実装するために代入演算子を「 \|=|」の
1種類だけにしました.
\cl や \javal 同様,代入演算の結果は代入した値になります.

\cmml では,代入演算子の左辺と右辺が厳密に同じ型でなければなりません.
これは,コンパイル時になるべく多くのバグを発見するための仕様です.
参照型の場合も型が厳密に一致している必要があります.
ただし,\|void[]| だけ例外的にどの参照型とも代入可能です.
自動的な型変換はありません.

\begin{mylist}
\begin{verbatim}
int     a;
boolean b = true;
char    c;
struct X { int r; };
struct Y { int r; };
a = 10;            // 同じ型なので代入可能
c = a;             // (エラー)型が異なるので代入できない
a = b;             // (エラー)型が異なるので代入できない
int i = a = 9;     // 代入演算(a=9)の結果は代入した値(9)
                   // 代入演算の結果(9)を i に代入する
X x = { 1 };
Y y;
y = x;             // (エラー)型が異なるので代入できない
void[] p = x;      // void[] にはどんな参照型も代入可能
y = p;             // void[] はどんな参照型にも代入可能
\end{verbatim}
\end{mylist}

\subsection{数値演算}

int 型データの計算に,2項演算子の
「\|+|」(和),
「\|-|」(差),
「\|*|」(積),
「\|/|」(商),
「\|%|」(余)
が使用できます.
その他に,単項演算子
「\|+|」,
「\|-|」が使用できます.
演算子の優先順位は数学と同じです.
計算(数値演算)をして,計算結果を変数に代入(代入演算)する例を次に示します.

\begin{mylist}
\begin{verbatim}
x = -10 + 3 * 2;
\end{verbatim}
\end{mylist}

\subsection{比較演算}

{\bf (1)整数型の大小比較と同値の判定},
{\bf (2)参照型,文字型,論理型の同値の判定}ができます.
大小比較の演算子は,
「\|>|」(より大きい),
「\|>=|」(以上),
「\|<|」(未満),
「\|<=|」(以下) の4種類です.
同値を判定する演算子は,
「\|==|」(等しい),
「\|!=|」(異なる)
の2種類です.
比較演算の結果は論理型です.
比較演算の結果を論理型変数に代入することができます.
論理型は if 文や while 文などの条件に使用できます.
次に,比較演算の例を示します.

\begin{mylist}
\begin{verbatim}
int x = 11;
boolean b;
b = x > 10;              // 整数の大小比較
if (b==false) { ... }    // 論理型の同値判定
\end{verbatim}
\end{mylist}

\subsection{論理演算}

論理型のデータを対象にした演算です.
演算結果も論理型になります.
単項演算子「\|!|」(否定),
2項演算子
「\|&&|」(論理積),
「\verb/||/」(論理和)が使用できます.
次に,論理演算の例を示します.
論理型変数 \|b| に比較結果を求めた後で,
\|b| の否定を if 文の条件に使用しています.

\begin{mylist}
\begin{verbatim}
int x = 11;
boolean b;
b = 10 <= x && x <= 20;  // (10<=x) と (x<=20) の論理積を b に代入
if (!b) { ... }          // 論理値の否定
\end{verbatim}
\end{mylist}

\subsection{ビット毎の論理演算}

整数値を対象にした演算です.
演算結果も整数値になります.
単項演算子「\|~|」(全ビットを反転),
2項演算子
「\|&|」(ビット毎の論理積),
「\verb/|/」(ビット毎の論理和),
「\|^|」(ビット毎の排他的論理和)が使用できます.
次に,ビット毎の論理演算の例を示します.
マスクを使用して,変数 \|x| の下位8ビットを取り出して表示します.
\|printf|の括弧内で下位8ビットを取り出すために
ビット毎の論理演算をしています.

\begin{mylist}
\begin{verbatim}
int x   = 0xabcd;
int msk = 0x00ff;
printf("%x", x & msk);
\end{verbatim}
\end{mylist}

\subsection{シフト演算}

整数値を対象にした演算です.
演算結果も整数値になります.
2項演算子
「\|>>|」(右算術シフト),
「\|<<|」(左算術シフト)が使用できます.
算術シフトしかありません.
次に,シフト演算の例を示します.
シフト演算とマスクを使用して,
変数 \|x| の値の上位8ビットを取り出して表示します.
算術シフトですから,マスクを忘れないように注意する必要があります.
\|printf|の括弧内で上位8ビットを取り出す計算をしています.

\begin{mylist}
\begin{verbatim}
int x   = 0xabcd;
printf("%x", x >> 8 & 0x00ff);
\end{verbatim}
\end{mylist}

\subsection{参照演算}

配列要素と構造体メンバをアクセスするための「\|[ 添字式 ]|」や「\|.|」は,
参照を対象にする演算子と考えることができます.
「\|[ 添字式 ]|」演算子は,配列参照と添字式から配列要素を求めます.
「\|.|」演算子は,構造体参照とメンバ名からメンバを求めます.
配列の配列である多次元配列のアクセスは,
「\|[ 添字式 ]|」演算子により取り出した配列要素が配列参照なので,
更に「\|[ 添字式 ]|」演算子により次の配列要素を取り出すと考えます.
実際,\cmm コンパイラの内部でもそのように考えて扱っています.
次に多次元配列や構造体を使用したプログラムの例を示します.
「\ref{chap3:ref} 参照型」に示したプログラム例も参考にしてください.

\begin{mylist}
\begin{verbatim}
// 多次元配列は配列参照の配列と考える
int[][] a = {{1,2},{3,4}};        // 2次元配列を作る
void f() {
  int[] b = a[0];                 // 2次元配列の要素は1次元配列
  int   c = b[1];                 // 1次元配列の要素は int 型
  int   d = a[0][1];              // c と d は同じ結果になる
}

// 構造体リスト例
struct List {                     // リスト構造のノード型
  List next;                      // 次のノードの参照
  int  val;                       // ノードの値
};
List a;                           // リストのルートを作る

void g() {
  a = malloc(sizeof(List));       // リストの先頭ノードを作る
  a.val  = 1;
  a.next = malloc(sizeof(List));  // リストの2番目ノードを作る
  a.next.val  = 2;
  a.next.next = null;             // 2番目ノードは参照の参照
}
\end{verbatim}
\end{mylist}

\subsection{sizeof 演算}

変数のサイズを知るための演算子です.
\|malloc| で領域を割り付けるとき使用します.
「\|sizeof(型)|」のように使用します.
型が基本型の場合は「変数の領域サイズ」,
構造体の場合は「インスタンスの領域サイズ」をバイト単位で返します.
型が配列型の場合は,何型の配列かとは関係なく「参照の領域サイズ
(アドレスのバイト数)」を返します.
通常,参照の領域サイズは「\|sizeof(void[])|」と書きます.
%配列インスタンスのサイズが必要なときは,
%\|sizeof|の値に配列の要素数を掛け算して求めます.
以下に使用例を示します.

\begin{mylist}
\begin{verbatim}
// sizeof 演算子の使用例
int a = sizeof(int);                // int のサイズ(TaC版で 2)
int b = sizeof(char);               // charのサイズ(TaC版で 1)
int c = sizeof(boolean);            // booleanのサイズ(TaC版で 1)
int d = sizeof(void[]);             // 参照のサイズ(TaC版で 2)
struct X { int x; int y; };
int e = sizeof(X);                  // 構造体Xのサイズ(TaC版で 4)
X[] f = rMalloc(3);                 // 大きさ3の参照配列を準備
f[0] = malloc(sizeof(X));           // 構造体インスタンスを割当
f[1] = malloc(sizeof(X));
f[2] = malloc(sizeof(X));
\end{verbatim}
\end{mylist}

\subsection{addrof 演算}
\label{chap3:addrof}

関数や大域変数のアドレスを知るための演算子です.
「\|addrof(大域名)|」のように使用し整数型の値を返します.
\|interrupt|関数を割込みベクタに登録したりする目的で使用します.
配列や構造体の要素や,関数のローカル変数のアドレスを求めることはできません.

\subsection{ord 演算}
\label{chap3:ord}

char型,boolean型の値をint型に変換します.
char型の場合は文字の ASCII コード,
boolean型の場合は\|true=1|,\|false=0|となります.
なお,v3.4.0以降のコンパイラでは,
char型からint型へは自動的な型変換が可能になり
char型を引数にするord演算子は不要になりましたが,
以前のバージョンとの互換性のために機能を残してあります.

\subsection{chr 演算}
\label{chap3:chr}

int型の ASCII コードから,char型の文字に変換します.

\subsection{bool 演算}
\label{chap3:bool}

int型の\|1|,\|0|から,boolean型の論理値に変換します.

以下に,ord,chr,bool 演算子の使用例を示します.

\begin{mylist}
\begin{verbatim}
// ord(), chr(), bool()演算子の使用例
int i = 0x41;        // 'A'のASCIIコード
char c = chr(i);     // cに,文字'A'が代入される
c = chr(ord(c)+1);   // cに,文字'B'が代入される
i = ord(true);       // i は 1 になる
boolean b = bool(1); // b は true になる
\end{verbatim}
\end{mylist}

\subsection{カンマ演算}
複数の式を接続して文法上一つの式にします.
例えば,
式が一つしか書けない for 文の再初期化部分に二つの式を書くために使用できます.
カンマ演算子は,最も優先順位の低い演算子です.

\begin{mylist}
\begin{verbatim}
// カンマ式を使用した例
for (i=0;i<0;i=i+1,j=j-1) { ... } // 再初期化に二つの式を書いた
\end{verbatim}
\end{mylist}

\subsection{演算子のまとめ}

関数の呼び出しも厳密には「\|(引数リスト)|」演算子と考えることができますが,
関数の呼び出しかたは\pageref{chap3:func}ページで説明したので省略します.
その他に,演算の順序を明確にするための「\|(式)|」も含めて,
演算子の優先順位を\tabref{chap3:operator}にまとめます.
表の上の方に書いてある演算子が優先順位の高い演算子です.
同じ高さにある演算子同士は同じ優先順位になります.
計算は優先順位の高いものから順に行われます.
例えば,「\|*|」は「\|+|」よりも優先順位が高いので先に計算されます.
優先順位が同じ場合は結合規則の欄で示した順に計算されます.

\begin{mytable}{tbhp}{演算子の優先順位}{chap3:operator}
\begin{tabular}{l|l}
\hline
\multicolumn{1}{c}{演算子} & \multicolumn{1}{|c}{結合規則} \\
\hline
\|sizeof()|,\|addrof()|,\|ord()|,\|chr()|,\|bool()|,
関数呼出,\|()|,\|[]|,\|.|                   & 左から右 \\
\|+|(単項演算子),\|-|(単項演算子),\|!|,\|~| & 右から左 \\
\|*|,\|/|,\|%|                               & 左から右 \\
\|+|,\|-|                                     & 左から右 \\
\|<<|,\|>>|                                   & 左から右 \\
\|>|,\|>=|,\|<|,\|<=|                       & 左から右 \\
\|==|,\|!=|                                   & 左から右 \\
\|&|                                           & 左から右 \\
\|^|                                           & 左から右 \\
\verb/|/                                       & 左から右 \\
\|&&|                                          & 左から右 \\
\verb;||;                                      & 左から右 \\
\|=|                                           & 右から左 \\
\|,|                                           & 左から右 \\
\end{tabular}
\end{mytable}

\section{文}

関数の内部に記述されるもので,変数宣言以外の記述を文と呼びます.
文は機械語に変換されて実行されます.
文には,空文,式文,ブロック,制御文(if 文や for 文)等があります.
\cmml の制御文には switch 文がありませんが
\cl にある他の制御文は一通り揃っています.

\subsection{空文}

単独の「\|;|」を空文と呼び,
文法上,一つの文として扱います.
本文のない for 文等で形式的な本文として使用します.
次に,空文を用いる例を示します.

\begin{mylist}
\begin{verbatim}
for (i=2; i<n; i=i*i)  // 必要なことはこの1行で全部記述できた
  ;                    // 空文
\end{verbatim}
\end{mylist}

\subsection{式文}

式の後ろに「\|;|」を付けたものを式文と呼び,
文法上,一つの文として扱います.
\cmml の文法に代入文はありませんが,
代入式に「\|;|」を付けた「式文」が同じ役割に使用されます.

\begin{quote}
\begin{verbatim}
式 ;
\end{verbatim}
\end{quote}

\subsection{ブロック}

「\|{|」と「 \|}|」で括って複数の文をグループ化し,
文法上,一つの文にします.
if 文や while 文の「本文」は,文法的には一つの文でなければなりません.
複数の文を「本文」として実行させたい場合はブロックにします.
また,ブロックはローカル変数の有効範囲を決定します.
ブロック内部で宣言された変数の有効範囲はブロックの最後までです.

\begin{quote}
\begin{verbatim}
{ 文 または 変数宣言 ... }
\end{verbatim}
\end{quote}

\subsection{if 文}

条件によって実行の流れを変更するための文です.
「条件式」は論理型の値を返す式でなければなりません.
「条件式」の値が true の場合「本文1」が実行され,
false の場合「本文2」が実行されます.
なお,else 節(「else 本文2」の部分)は省略することができます.

\begin{quote}
\begin{verbatim}
if ( 条件式 ) 本文1 【 else 本文2 】 
\end{verbatim}
\end{quote}

\subsection{while 文}

条件が成立している間,while 文の「本文」を実行します.
「条件式」は論理型の値を返す式でなければなりません.
まず,「条件式」を計算し,値が true なら「本文」が実行されます.
これは,「条件式」の値が false になるまで繰り返されます.

\begin{quote}
\begin{verbatim}
while ( 条件式 ) 本文
\end{verbatim}
\end{quote}

\subsection{do-while 文}

条件が成立している間,do-while 文の「本文」を実行します.
「条件式」は論理型の値を返す式でなければなりません.
まず「本文」を実行し,次に「条件式」を計算します.
「条件式」の値が true なら,再度,「本文」の実行に戻ります.
これは,「条件式」の値が false になるまで繰り返されます.

\begin{quote}
\begin{verbatim}
do 本文 while ( 条件式 ) ;
\end{verbatim}
\end{quote}

\subsection{for 文}

便利に拡張された while 文です.
「条件式」は論理型でなければなりません.
まず,「初期化式」か「ローカル変数宣言」を実行します.
次に「条件式」を計算し,値が true なら「本文」を実行します.
最後に「再初期化式」を実行し,その後「条件式」の計算に戻ります.
これは,「条件式」の値が false になるまで繰り返されます.

「ローカル変数宣言」で宣言された変数は,「条件式」,「再初期化式」,
「本文」で使用することができますが,それ以降では使用できません.
「初期化式」,「条件式」,「再初期化式」のどれも省略可能です.
「条件式」を省略した場合は無限ループの記述になります.

\begin{quote}
\begin{verbatim}
for(【初期化式|ローカル変数宣言】;【条件式】;【再初期化式】) 本文
\end{verbatim}
\end{quote}

\begin{mylist}
\begin{verbatim}
// for 文の使用例
for (int j=0; j<10; j=j+1) {
  ...
  if (j==5) { ...                   // j が使用できる
  ...
}
n = j;                              // (エラー)j が未定義になる

for (;;) { ... }                    // 無限に本文を繰り返す
\end{verbatim}
\end{mylist}

\subsection{return 文}

関数から戻るときに使用します.
関数の途中で使用すると,関数の途中から呼び出し側に戻ることができます.
void 型以外の関数では,関数の最後の文が return 文でなければなりません.
「式」は void 型の関数では書いてはなりません.
逆に,void 型以外の関数では書かなければなりません.
「式」の型と関数の型は一致していなければなりません.
なお,interrupt 関数には void 型の関数と同じルールが適用されます.

\begin{quote}
\begin{verbatim}
return 【 式 】 ;
\end{verbatim}
\end{quote}

\begin{mylist}
\begin{verbatim}
// void 型以外の関数
int f() {
  ...
  if (err) return 1;     // f の途中から戻る
  ...
  return 0;              // この return は省略できない
}

// void 型の関数
void g() {
  ...
  if (err) return;       // g の途中から戻る
  ...
  return;                // この return は省略しても良い
}
\end{verbatim}
\end{mylist}

\subsection{break 文}

for 文や while 文,do-while 文の繰り返しから脱出します.
多重ループから一度に脱出することはできません.

\begin{quote}
\begin{verbatim}
break;
\end{verbatim}
\end{quote}

\subsection{continue 文}

for 文や while 文,do-while 文の本文の実行をスキップします.
for 文では再初期化式に,while 文と do-while 文では条件式にジャンプします.

\begin{quote}
\begin{verbatim}
continue;
\end{verbatim}
\end{quote}
 % C-- 言語の仕様
%
%  4章 ライブラリ関数
%
\chapter{ライブラリ関数}

\cmml で使用できる関数です.
必要最低限の関数が,{\tacos}版,{\cl}トランスレータ版で使用できます.

\section{標準入出力ライブラリ}
\label{chap4:stdio}

\|#include <stdio.hmm>|を書いた後で使用します.
トランスレータ版では{\cl}の高水準I/O関数の呼出しに変換されます\footnote{
単純に置き換えができる場合は同じ名前の{\cl}関数を呼び出します.
そうでない場合もあります.}.
{\tacos}版でも入出力の自動的なバッファリングを行います.
{\tacos}版ではバッファサイズは 128 バイトです.
以下の関数が使用できます.

\subsection{printf 関数}

標準出力ストリームに\|format|文字列を用いた変換付きで出力します.
出力した文字数を関数の値として返します.

\begin{quote}
\begin{verbatim}
#include <stdio.hmm>
public int printf(char[] format, ...);
\end{verbatim}
\end{quote}

\|format|文字列に以下のような変換を記述できます.

\begin{quote}
\begin{verbatim}
%[-][数値]変換文字
\end{verbatim}
\end{quote}

\|-|を書くと左詰めで表示します.
数値は表示に使用するカラム数を表します.
数値を\|0|で開始した場合は,
数値の右づめ表示で空白の代わりに\|0|が用いられます.
使用できる変換文字は次の表の通りです.

\begin{quote}
\begin{tabular}{c|l}
\multicolumn{1}{c|}{変換文字} & \multicolumn{1}{c}{意味} \\\hline
\|o| & 整数値を8進数で表示する \\
\|d| & 整数値を10進数で表示する \\
\|x| & 整数値を16進数で表示する \\
\|c| & ASCIIコードに対応する文字を表示する \\
\|s| & 文字列を表示する \\
\|%| & \|%|を表示する \\
\|ld|& \|int|配列で表現した32ビット符号なし整数値を10進数で表示する \\
\end{tabular}
\end{quote}

\subsection{puts関数}

標準出力ストリームへ1行出力します.
エラーが発生した場合は\|true|を,正常時には\|false|を返します.

\begin{quote}
\begin{verbatim}
#include <stdio.hmm>
public boolean puts(char[] s);
\end{verbatim}
\end{quote}

\subsection{putchar関数}

標準出力ストリームへ1文字出力します.
エラーが発生した場合は\|true|を,正常時には\|false|を返します.

\begin{quote}
\begin{verbatim}
#include <stdio.hmm>
public boolean putchar(char c);
\end{verbatim}
\end{quote}

\subsection{getchar関数}

標準入力ストリームから1文字入力します.
\cl の\|getchar|関数と異なり\|char|型なので EOF チェックができません.
現在のところ,\tacos では標準入力を EOF にする方法は準備されていません.

\begin{quote}
\begin{verbatim}
#include <stdio.hmm>
public char getchar();
\end{verbatim}
\end{quote}

\subsection{fopen関数}

ファイルを開きます.
\|path|はファイルへのパス,
\|mode|はオープンのモードです.
パスは``\|/|''区切りで表現します.
%\tacos にはカレントディレクトリはありません.
\|fopen|は正常時に\|FILE|構造体,
エラー時に\|null|を返します.

\begin{quote}
\begin{verbatim}
#include <stdio.hmm>
public FILE fopen(char[] path, char[] mode);
\end{verbatim}
\end{quote}

{\tacos}版では\|mode|が次のような意味を持ちます.
なお,トランスレータ版では,\|mode|は{\cl}の\|fopen|にそのまま渡されます.

\begin{quote}
\begin{tabular}{c|l}
\multicolumn{1}{c|}{mode} & \multicolumn{1}{c}{意味} \\\hline
\|"r"| & 読み込みモードで開く \\
\|"w"| & 書き込みモードで開く(ファイルが無ければ作る) \\
\|"a"| & 追記モードで開く(ファイルが無ければ作る)
\end{tabular}
\end{quote}

%\|fopen|は,
%書き込みモードか追記モード時にファイルが存在しない場合は,
%自動的にファイルを作成します.

\subsection{fclose関数}

ストリームをクローズします.
\tacos では,
標準入出力ストリーム(\|stdin|,\|stdout|,\|stderr|)を
クローズすることはできません.
\|fclose|は正常時に\|false|,
エラー時に\|true|を返します.

\begin{quote}
\begin{verbatim}
#include <stdio.hmm>
public boolean fclose(FILE stream);
\end{verbatim}
\end{quote}

\subsection{fseek関数}

\|stream|で指定したファイルの読み書き位置を変更します.
seek 位置は,
{\tacos}版では\|offsh|(16bit),
\|offsl|(16bit)を合わせた32bitで指定します.
トランスレータ版では\|offsh|(32bit),
\|offsl|(32bit)を合わせた64bitで指定します.

正常時は\|false|を返します.
エラーが発生した場合,{\tacos}版ではプログラムを終了します.
トランスレータ版では\|true|を返します.

\begin{quote}
\begin{verbatim}
#include <stdio.hmm>
public boolean fseek(FILE stream, int offsh, int offsl);
\end{verbatim}
\end{quote}

\subsection{fsize関数}

\|path|で指定したファイルのサイズを\|size|に書き込みます.
ファイルサイズは,
{\tacos}版では\|size[0]|(上位16bit),
\|size[1]|(下位16bit)を合わせた32bitです.
トランスレータ版では\|size[0]|(上位32bit),
\|size[1]|(下位32bit)を合わせた64bitです.

正常時は\|false|を返します.
エラーが発生した場合,{\tacos}版ではプログラムを終了します.
トランスレータ版では\|true|を返します.

\begin{quote}
\begin{verbatim}
#include <stdio.hmm>
public boolean fsize(char[] path, int[] size);
\end{verbatim}
\end{quote}

\subsection{fprintf関数}

出力ストリームを明示できる\|printf|関数です.
\|stream|に出力先を指定します.
出力ストリームは,\|fopen|で開いたファイルか\|stdout|,
\|stderr|です.

\begin{quote}
\begin{verbatim}
#include <stdio.hmm>
public int fprintf(FILE stream, char[] format, ...);
\end{verbatim}
\end{quote}

\subsection{fputs関数}

出力ストリームを明示できる\|puts|関数です.
\|stream|に出力先を指定します.
出力ストリームは,\|fopen|で開いたファイルか\|stdout|,
\|stderr|です.

\begin{quote}
\begin{verbatim}
#include <stdio.hmm>
public boolean fputs(char[] s, FILE stream);
\end{verbatim}
\end{quote}

\subsection{fputc関数}

出力ストリームを明示できる\|putchar|関数です.
\|stream|に出力先を指定します.
出力ストリームは,\|fopen|で開いたファイルか\|stdout|,
\|stderr|です.

\begin{quote}
\begin{verbatim}
#include <stdio.hmm>
public boolean fputc(char c, FILE stream);
\end{verbatim}
\end{quote}

\subsection{fgets関数}

任意の入力ストリームから1行入力します.
入力は\|buf|に文字列として格納します.
\|n|には\|buf|のサイズを渡します.
通常,\|buf|に\|'\n'|も格納されます.
\|fgets|は,EOFで\|null|を,
正常時には\|buf|を返します.

\begin{quote}
\begin{verbatim}
#include <stdio.hmm>
public char[] fgets(char[] buf, int n, FILE stream);
\end{verbatim}
\end{quote}

\cmm では,\cl の\|gets|が使用できません.
\|gets|はバッファオーバーフローの危険があるので\cmm には持込みませんでした.
\cmm で,\|gets|を使用したい時は\|fgets|を使用して次のように書きます.

\begin{quote}
\begin{verbatim}
while (fgets(buf, N, stdin)!=null) { ...
\end{verbatim}
\end{quote}


\subsection{fgetc関数}

任意の入力ストリームから1文字入力します.
\cl の\|fgetc|関数と異なり\|char|型なので EOF チェックができません.
EOFチェックは\|feof|関数を用いて行います.

\begin{quote}
\begin{verbatim}
#include <stdio.hmm>
public char fgetc(FILE stream);
\end{verbatim}
\end{quote}

{\tacos}版では安全のため,
\|fgetc|関数がEOFに出会うと強制的にプログラムを終了する仕様になっています.
\|fgetc|関数を実行する前に,必ず,
\|feof|関数を用いて EOF チェックをする必要があります.
\pageref{chap4:cat1}ページのソースコードに使用例があります.

\subsection{feof関数}

入力ストリームが EOF になっていると \|true| を返します.
\|fgetc| を実行する前に EOF チェックのために使用します.
{\bf \cl の\|feof|関数と仕様が異なります.}
\cl の\|feof|関数はストリームが EOF になった後で \|true| になりますが,
\cmml の\|feof|関数は次回の操作で EOF になるタイミングで \|true| に
なります.

\begin{quote}
\begin{verbatim}
#include <stdio.hmm>
public boolean feof(FILE stream);
\end{verbatim}
\end{quote}

\subsection{ferror関数}

ストリームがエラーを起こしていると \|true| を返します.

\begin{quote}
\begin{verbatim}
#include <stdio.hmm>
public boolean ferror(FILE stream);
\end{verbatim}
\end{quote}

\subsection{fflush関数}

出力ストリームのバッファをフラッシュします.
入力ストリームをフラッシュすることはできません.
正常時\|false|,エラー時\|true| を返します.
\|stderr|はバッファリングされていないので,
フラッシュしても何も起きません.

\begin{quote}
\begin{verbatim}
#include <stdio.hmm>
public boolean fflush(FILE stream);
\end{verbatim}
\end{quote}

\subsection{readDir関数}
\label{chap4:readDir}

FAT16ファイルシステムのディレクトリファイルを読みます.
\|fd|には\|open|システムコールでオープン済のファイル記述子を,
\|dir|には\|Dir|構造体のインスタンスを渡します.
\|Dir|構造体の\|name|メンバーは,
大きさ12の文字配列で初期化されている必要があります.

\begin{quote}
\begin{verbatim}
#include <stdio.hmm>
public int readDir(int fd, Dir dir);
\end{verbatim}
\end{quote}

\|Dir|構造体は\|stdio.hmm|中で次のように宣言されています.

\begin{quote}
\begin{verbatim}
struct Dir {
    char[] name;              // ファイル名
    int    attr;              // ファイルの属性
    int    clst;              // ファイルの開始クラスタ
    int    lenH, lenL;        // ファイルの長さ
};
\end{verbatim}
\end{quote}

次に\|ls|プログラムのソースコードから抜粋したreadDir関数の使用例を示します.

\begin{mylist}
\begin{verbatim}
#include <stdio.hmm>
Dir dir = {"            ", 0, 0, 0, 0 };
int[] fLen = array(2);

// ディレクトリの一覧を表示する
int printDir(char[] fname) {
  int fd = open(fname, READ);                // ディレクトリを開く
  if (fd<0) {
    perror(fname);
    return 1;
  }

  printf("FileNameExt Attr Clst FileLength\n");
  while (readDir(fd, dir)>0) {               // ファイルが続く間
    fLen[0]=dir.lenH;
    fLen[1]=dir.lenL;
    printf("%11s 0x%02x %4d %9ld\n",         // ファイルの一覧出力
            dir.name, dir.attr, dir.clst,fLen);
  }
  close(fd);
  return 0;
}
\end{verbatim}
\end{mylist}

\subsection{perror関数}

\|errno|グローバル変数の値に応じたエラーメッセージを表示します.
\|msg|はエラーメッセージの先頭に付け加えます.
\|errno|にはシステムコールやライブラリ関数がエラー番号をセットします.
\tabref{chap4:err}に{\tacos}版のエラーとメッセージの一覧を示します.
\|errno|グローバル変数と表中の記号名は\|errno.hmm|で宣言されています.

\begin{quote}
\begin{verbatim}
#include <stdio.hmm>
#include <errno.hmm>
public int errno;
public void perror(char[] msg);
\end{verbatim}
\end{quote}

\begin{mytable}{tbhp}{エラー一覧}{chap4:err}
\begin{tabular}{l|l|l}
\multicolumn{1}{c|}{記号名}
 & \multicolumn{1}{c|}{メッセージ}
 & \multicolumn{1}{c}{意味} \\\hline
ENAME     & Invalid file name           & ファイル名が不正 \\
ENOENT    & No such file or directrory  & ファイルが存在しない \\
EEXIST    & File exists                 & 同名ファイルが存在する \\
EOPEND    & File is opened              & 既にオープンされている \\
ENFILE    & File table overflow         & システム全体のオープン数超過 \\
EBADF     & Bad file number             & ファイル記述子が不正 \\
ENOSPC    & No space left on device     & デバイスに空き領域が不足 \\
EPATH     & Bad path                    & パスが不正 \\
EMODE     & Bad mode                    & モードが一致しない \\
EFATTR    & Bad attribute               & ファイルの属性が不正 \\
ENOTEMP   & Directory is not empty      & ディレクトリが空でない \\
EINVAL    & Invalid argument            & 引数が不正 \\
EMPROC    & Process table overflow      & プロセスが多すぎる \\
ENOEXEC   & Bad EXE file                & EXE ファイルが不正 \\
EMAGIC    & Bad MAGIC number            & 不正なマジック番号 \\
EMFILE    & Too many open files         & プロセス毎のオープン数超過 \\
ECHILD    & No children                 & 子プロセスが存在しない \\
ENOZOMBI  & No zombie children          & ゾンビ状態の子が存在しない \\
ENOMEM    & Not enough memory           & 十分な空き領域が無い \\
EAGAIN    & Try again                   & 再実行が必要 \\
ESYSNUM   & Invalid system call number  & システムコール番号が不正 \\
EZERODIV  & Zero division               & ゼロ割り算 \\
EPRIVVIO  & Privilege violation         & 特権違反 \\
EILLINST  & Illegal instruction         & 不正命令 \\
EMEMVIO   & Memory violation            & メモリ保護違反 \\
EUSTK     & Stack overflow              & スタックオーバーフロー \\
EUMODE    & stdio: Bad open mode        & モードと使用方法が矛盾 \\
EUBADF    & stdio: Bad file pointer     & 不正な fp が使用された \\
EUEOF     & fgetc: EOF was ignored      & fgetc前にEOFチェック必要 \\
EUNFILE   & fopen: Too many open files  & プロセス毎のオープン超過 \\
EUSTDIO   & fclose: Standard i/o should & 標準ioはクローズできない \\
          &  not be closed              &                          \\
EUFMT     & fprintf: Invalid conversion & 書式文字列に不正な変換 \\
EUNOMEM   & malloc: Insufficient memory & ヒープ領域が不足 \\
EUBADA    & free: Bad address           & mallocした領域ではない \\
\end{tabular}
\end{mytable}

\subsection{プログラム例}

\cmml で記述した,標準入出力関数の使用例を以下に示します.

\subsubsection{TacOS 専用のプログラム例}

\begin{itemize}
\item エラー処理 \\
\tacos では,\|errno|変数にセットされるエラー番号が負の値になっています.
また,アプリケーションが負の終了コードで終わった場合,
シェルが終了コードを\|errno|とみなし
エラーメッセージを表示する仕様になっています.
更にライブラリは,ユーザプログラムのバグが原因と考えられるエラーや,
メモリ不足のような対処が難しいエラーが発生したとき,
負の終了コードでプログラムを終了します.
そこで,以下のようなエラー処理を簡略化したプログラムを書くことができます.

このプログラムは,
メモリ不足で\|FILE|構造体の割り付けができないなど,
対処が難しいエラーの場合に,
\|fopen|内部で自動的に\|errno|を終了コードにして終了します.
プログラムの終了コードによりシェルがエラーメッセージを表示します.

ファイルが見つからないなどプログラムに知らせた方が良いエラーの場合は,
\|fopen|がエラーを示す戻り値(\|null|)を持って返ります.
可能ならユーザプログラムがエラー回復を試みるべきです.
下のプログラムはエラー回復を試みることなく
\|errno|を終了コードとして終了しています.
エラーメッセージの表示をプログラム中で行っていませんが,
シェルが\|errno|に対応したエラーメッセージを表示します.

\item EOFの検出 \\
EOF の検出は\|feof|関数を用いて行います.
{\cl}のプログラムと書き方が異なりますので注意してください.

\end{itemize}

\begin{mylist}
\begin{verbatim}
// ファイルの内容を表示するプログラム(TacOS 専用バージョン)
#include <stdio.hmm>
#include <errno.hmm>
public int main(int argc, char[][] argv) {
  FILE fp = fopen("a.txt", "r");
  if (fp==null) exit(errno);     // エラー表示をシェルに任せる
  while (!feof(fp)) {
    putchar(fgetc(fp));
  }
  fclose(fp);
  return 0;
}
\end{verbatim}
\label{chap4:cat1}
\end{mylist}

\subsubsection{TacOS トランスレータ共通版のプログラム例}

\cl プログラム風に記述することもできます.
前の例ではシェルがエラーメッセージを表示したので,
エラーメッセージの内容をプログラムから細かく指定することができませんでした.
次の例ではプログラムが自力でエラーメッセージを表示するので,
エラーになったファイルの名前をエラーメッセージに含めることができます.

エラー表示を行ったプログラムは終了コード\|1|で終わります.
終了コードが正なので,シェルはエラーメッセージを表示しません.

\begin{mylist}
\begin{verbatim}
// ファイルの内容を表示するプログラム
// (トランスレータ,TacOS 共通バージョン)
#include <stdio.hmm>
public int main(int argc, char[][] argv) {
  char fname = "a.txt";
  FILE fp = fopen(fname, "r");
  if (fp==null) {
    perror(fname);     // エラー表示を自分で行う
    return 1;
  }
  while (!feof(fp)) {
    putchar(fgetc(fp));
  }
  fclose(fp);
  return 0;
}
\end{verbatim}
\end{mylist}

\section{標準ライブラリ}

\|#include <stdlib.hmm>|を書いた後で使用します.

\subsection{malloc関数}

ヒープ領域に\|size|バイトのメモリ領域を確保し,
領域を指す参照を返します.
\|malloc|関数は\|void[]|型なので,
領域を指す参照は全ての参照変数に代入できます.
\|malloc|関数は構造体領域の確保に使用します.
配列領域は,\|iMalloc|,\|cMalloc|,\|bMalloc|,\|rMalloc|を用いて確保します.
配列領域には添字チェックのためのデータが組み込まれますので,
\|malloc|関数で割り当てることはできません.

\begin{quote}
\begin{verbatim}
#include <stdlib.hmm>
public void[] malloc(int size);
\end{verbatim}
\end{quote}

{\tacos}版では,ヒープ領域に十分な空きが見つからないとき
終了コード\|EUNOMEM|でプログラムを終了します.
トランスレータ版では,エラーメッセージを表示したあと
終了コード\|1|でプログラムを終了します.

\subsection{calloc関数}

連続したヒープ領域に\|s|バイトのメモリ領域を\|c|個確保し,
領域を指す参照を返します.
確保した領域はゼロでクリアします.
エラー処理と制約は\|malloc|関数と同様です.

\begin{quote}
\begin{verbatim}
#include <stdlib.hmm>
public void[] calloc(int c, int s);
\end{verbatim}
\end{quote}

\subsection{iMalloc関数}

ヒープ領域にint型の要素数\|c|の配列領域を確保し,
領域を指す参照を返します.
\|iMalloc|関数は\|int[]|型なので,
int型配列の参照変数にしか代入できません.
実行時の添字範囲チェックに必要なデータも作成するので,
\|malloc|関数で確保した領域と内容が異なります.

\begin{quote}
\begin{verbatim}
#include <stdlib.hmm>
public int[] iMalloc(int c);
\end{verbatim}
\end{quote}

\subsection{cMalloc関数}

ヒープ領域にchar型の要素数\|c|の配列領域を確保し,
領域を指す参照を返します.
\|cMalloc|関数は\|char[]|型なので,
char型配列の参照変数にしか代入できません. 
実行時の添字範囲チェックに必要なデータも作成するので,
\|malloc|関数で確保した領域と内容が異なります.

\begin{quote}
\begin{verbatim}
#include <stdlib.hmm>
public char[] cMalloc(int c);
\end{verbatim}
\end{quote}

\subsection{bMalloc関数}

ヒープ領域にboolean型の要素数\|c|の配列領域を確保し,
領域を指す参照を返します.
\|bMalloc|関数は\|boolean[]|型なので,
boolean型配列の参照変数にしか代入できません.
実行時の添字範囲チェックに必要なデータも作成するので,
\|malloc|関数で確保した領域と内容が異なります.

\begin{quote}
\begin{verbatim}
#include <stdlib.hmm>
public boolean[] bMalloc(int c);
\end{verbatim}
\end{quote}

\subsection{rMalloc関数}

ヒープ領域に参照型の要素数\|c|の配列領域を確保し,
領域を指す参照を返します.
\|rMalloc|関数は\|void[][]|型なので,
参照型配列の参照変数にしか代入できません.
実行時の添字範囲チェックに必要なデータも作成するので,
\|malloc|関数で確保した領域と内容が異なります.

\begin{quote}
\begin{verbatim}
#include <stdlib.hmm>
public void[][] rMalloc(int c);
\end{verbatim}
\end{quote}

\subsection{free関数}

\|malloc|関数や\|iMalloc|関数等で割当てた領域を解放します.
{\tacos}版では,領域がこれらの関数で割当てたものではない可能性がある場合
(マジックナンバーが破壊されている,管理されている空き領域と重なる等),
終了コード\|EUBADA|でプログラムを終了します.

\begin{quote}
\begin{verbatim}
#include <stdlib.hmm>
public void free(void[] mem);
\end{verbatim}
\end{quote}

\subsection{atoi関数}

\|atoi|関数は引数に渡した10進数文字列を解析して,
それが表現する値を返します.

\begin{quote}
\begin{verbatim}
#include <stdlib.hmm>
public int atoi(char[] s);
\end{verbatim}
\end{quote}

\subsection{htoi関数}

\|htoi|関数は引数に渡した16進数文字列を解析して,
それが表現する値を返します.

\begin{quote}
\begin{verbatim}
#include <stdlib.hmm>
public int htoi(char[] s);
\end{verbatim}
\end{quote}

\subsection{srand関数}

\|srand|関数は擬似乱数発生器を\|seed|で初期化します.

\begin{quote}
\begin{verbatim}
#include <stdlib.hmm>
public void srand(int seed);
\end{verbatim}
\end{quote}

\subsection{rand関数}

\|rand|関数は次の擬似乱数を発生します.

\begin{quote}
\begin{verbatim}
#include <stdlib.hmm>
public int rand();
\end{verbatim}
\end{quote}

\subsection{exit関数}

\|exit|関数はオープン済みのストリームをフラッシュしてから
プログラムを終了します.
\|status|は,親プロセスに返す終了コードです.
\|0|が正常終了の意味,\|1|以上はユーザが決めた終了コードです.

{\tacos}版では,負の終了コードが使用できます.
使用できるコードは\tabref{chap4:err}に記号名として定義されています.
負の値を返すと親プロセスがシェルの場合,
シェル側でエラーメッセージを表示してくれます.

\begin{quote}
\begin{verbatim}
#include <stdlib.hmm>
public void exit(int status);
\end{verbatim}
\end{quote}

\subsection{environ変数}

\|environ|変数は環境変数の配列です.配列の最後には\|null|が入ります.
\|setEnv|関数などにより更新することができます.
\|environ|変数は\|stdlib.hmm|中で宣言されます.
トランスレータ版との互換性のため自分で宣言しないで下さい.

\begin{quote}
\begin{verbatim}
#include <stdlib.hmm>
public char[][] environ;
\end{verbatim}
\end{quote}

\subsection{getEnv関数}

\|getEnv|関数は環境変数から値を取得します.
指定された環境変数が存在しない場合には\|null|を返します.

\begin{quote}
\begin{verbatim}
#include <stdlib.hmm>
public char[] getEnv(char[] name);
\end{verbatim}
\end{quote}

\subsection{putEnv関数}

\|putEnv|関数は環境変数を設定します.
\|str|は\|"name=value"|の形の文字列です.
エラーが発生した場合には\|true|を,正常時には\|false|を返します.

\begin{quote}
\begin{verbatim}
#include <stdlib.hmm>
public boolean putEnv(char[] str);
\end{verbatim}
\end{quote}

\subsection{setEnv関数}

\|setEnv|関数は環境変数を設定します.
\|overwrite|が\|false|のとき,
指定された環境変数がすでに存在すれば何もせず正常終了します.
エラーが発生した場合には\|true|を,正常時には\|false|を返します.

\begin{quote}
\begin{verbatim}
#include <stdlib.hmm>
public boolean setEnv(char[] name, char[] value, boolean overwrite);
\end{verbatim}
\end{quote}

\subsection{unsetEnv関数}

\|unsetEnv|関数は環境変数を削除します.
指定された環境変数が存在しない場合,何もせず正常終了します.
エラーが発生した場合には\|true|を,正常時には\|false|を返します.

\begin{quote}
\begin{verbatim}
#include <stdlib.hmm>
public boolean unsetEnv(char[] name);
\end{verbatim}
\end{quote}

\subsection{absPath関数}

\|absPath|関数はカレントディレクトリからの相対パスを絶対パスに変換します.
\|path|は相対パス,\|buf|は絶対パスを格納するバッファ,\|bufSiz|は\|buf|のサイズです.
エラーが発生した場合には\|true|を,正常時には\|false|を返します.

\begin{quote}
\begin{verbatim}
#include <stdlib.hmm>
public boolean absPath(char[] path, char[] buf, int bufSiz);
\end{verbatim}
\end{quote}

\subsection{getWd関数}

\|getWd|関数はカレントディレクトリを取得します.
\|getWd|は正常ならカレントディレクトリの文字列,エラー発生なら null を返します.
返り値の文字列を変更しないでください.

\begin{quote}
\begin{verbatim}
#include <stdlib.hmm>
public char[] getWd();
\end{verbatim}
\end{quote}

\subsection{chDir関数}

\|chDir|関数はカレントディレクトリを変更します.
\|pathname|は変更先のディレクトリです.
エラーが発生した場合には true を,正常時には false を返します.

\begin{quote}
\begin{verbatim}
#include <stdlib.hmm>
public boolean chDir(char[] pathname);
\end{verbatim}
\end{quote}

\section{文字列操作関数}

\|#include <string.hmm>|を書いた後で使用します.

\subsection{strCpy関数}

文字列\|s|を文字配列\|d|にコピーし,
\|d|を関数値として返します.

\begin{quote}
\begin{verbatim}
#include <string.hmm>
public char[] strCpy(char[] d, char[] s);
\end{verbatim}
\end{quote}

\subsection{strNcpy関数}

文字列\|s|の最大\|n|文字を文字配列\|d|にコピーし,
\|d|を関数値として返します.
文字配列の使用されない部分には\|'\0'|が書き込まれます.
文字列\|s|の長さが\|n|以上の場合は,
\|'\0'|が書き込まれないので注意して下さい.

\begin{quote}
\begin{verbatim}
#include <string.hmm>
public char[] strNcpy(char[] d, char[] s, int n);
\end{verbatim}
\end{quote}

\subsection{strCat関数}

文字列\|s|を文字配列\|d|に格納されている文字列の後ろに追加し,
\|d|を関数値として返します.

\begin{quote}
\begin{verbatim}
#include <string.hmm>
public char[] strCat(char[] d, char[] s);
\end{verbatim}
\end{quote}

\subsection{strNcat関数}

文字列\|s|の先頭\|n|文字未満を,
文字配列\|d|に格納されている文字列の後ろに追加し,
\|d|を関数値として返します.
\|d|に格納された文字列の最後には\|'\0'|が書き込まれます.

\begin{quote}
\begin{verbatim}
#include <string.hmm>
public char[] strNcat(char[] d, char[] s, int n);
\end{verbatim}
\end{quote}

\subsection{strCmp関数}

文字列\|s1|と文字列\|s2|を比較します.
\|strCmp|関数は,アスキーコード順で\|s1|が大きいとき正の値,
\|s1|が小さいとき負の値,同じ時\|0|を返します.

\begin{quote}
\begin{verbatim}
#include <string.hmm>
public int strCmp(char[] s1, char[] s2);
\end{verbatim}
\end{quote}

\subsection{strNcmp関数}

文字列\|s1|と文字列\|s2|の先頭\|n|文字を比較します.
\|strNcmp|関数は,
\|strcmp|関数同様にアスキーコード順で大小を判断します.

\begin{quote}
\begin{verbatim}
#include <string.hmm>
public int strNcmp(char[] d, char[] s, int n);
\end{verbatim}
\end{quote}

\subsection{strLen関数}

文字列\|s|の長さを返します.
長さに\|'\0'|は含まれません.

\begin{quote}
\begin{verbatim}
#include <string.hmm>
public int strLen(char[] s);
\end{verbatim}
\end{quote}

\subsection{strChr関数}

文字列\|s|の中で最初に文字\|c|が現れる位置を,
{\bf\|s|文字配列の添字}で返します.
文字\|c|が含まれていない場合は\|-1|を返します.

\begin{quote}
\begin{verbatim}
#include <string.hmm>
public int strChr(char[] s, char c);
\end{verbatim}
\end{quote}

\subsection{strRchr関数}

文字列\|s|の中で最後に文字\|c|が現れる位置を,
{\bf\|s|文字配列の添字}で返します.
文字\|c|が含まれていない場合は\|-1|を返します.

\begin{quote}
\begin{verbatim}
#include <string.hmm>
public int strRchr(char[] s, char c);
\end{verbatim}
\end{quote}

\subsection{strStr関数}

文字列\|s1|の中に文字列\|s2|が現れる位置を,
{\bf\|s1|文字配列の添字}で返します.
文字列\|s2|が含まれていない場合は\|-1|を返します.

\begin{quote}
\begin{verbatim}
#include <string.hmm>
public int strStr(char[] s1, char[] s2);
\end{verbatim}
\end{quote}

\section{文字クラス分類関数}

\|#include <ctype.hmm>|を書いた後で使用します.

\subsection{isAlpha関数}

文字\|c|がアルファベット('A'〜'Z','a'〜'z')なら\|true|を返します.

\begin{quote}
\begin{verbatim}
#include <ctype.hmm>
public boolean isAlpha(char c);
\end{verbatim}
\end{quote}

\subsection{isDigit関数}

文字\|c|が数字('0'〜'9')なら\|true|を返します.

\begin{quote}
\begin{verbatim}
#include <ctype.hmm>
public boolean isDigit(char c);
\end{verbatim}
\end{quote}

\subsection{isAlnum関数}

文字\|c|がアルファベットか数字
('A'〜'Z','a'〜'z','0'〜'9')なら\|true|を返します.

\begin{quote}
\begin{verbatim}
#include <ctype.hmm>
public boolean isAlnum(char c);
\end{verbatim}
\end{quote}

\subsection{isPrint関数}

文字\|c|が印刷可能文字(文字コード\|0x20|〜\|0x7e|の範囲)なら
\|true|を返します.

\begin{quote}
\begin{verbatim}
#include <ctype.hmm>
public boolean isPrint(char c);
\end{verbatim}
\end{quote}

\subsection{isLower関数}

文字\|c|がアルファベット小文字('a'〜'z')なら\|true|を返します.

\begin{quote}
\begin{verbatim}
#include <ctype.hmm>
public boolean isLower(char c);
\end{verbatim}
\end{quote}

\subsection{isUpper関数}

文字\|c|がアルファベット大文字('A'〜'Z')なら\|true|を返します.

\begin{quote}
\begin{verbatim}
#include <ctype.hmm>
public boolean isUpper(char c);
\end{verbatim}
\end{quote}

\subsection{isXdigit関数}

文字\|c|が16進数文字('0'〜'9','A'〜'F','a'〜'f')なら\|true|を返します.

\begin{quote}
\begin{verbatim}
#include <ctype.hmm>
public boolean isXdigit(char c);
\end{verbatim}
\end{quote}

\subsection{isSpace関数}

文字\|c|が空白文字(\|'\t'(TAB)|,\|'\n'(LF)|,\|'\x0b'(VT)|,
\|'\x0c'(FF)|,\|'\r'(CR)|,\|' '|)なら\|true|を返します.

\begin{quote}
\begin{verbatim}
#include <ctype.hmm>
public boolean isSpace(char c);
\end{verbatim}
\end{quote}

\subsection{toLower関数}

文字\|c|がアルファベット大文字なら小文字に変換して返します.
文字\|c|がアルファベット大文字以外の場合は変換しないで返します.

\begin{quote}
\begin{verbatim}
#include <ctype.hmm>
public char toLower(char c);
\end{verbatim}
\end{quote}

\subsection{toUpper関数}

文字\|c|がアルファベット小文字なら大文字に変換して返します.
文字\|c|がアルファベット小文字以外の場合は変換しないで返します.

\begin{quote}
\begin{verbatim}
#include <ctype.hmm>
public char toUpper(char c);
\end{verbatim}
\end{quote}

\section{特殊な関数}

\cmml にはキャスティング演算や,ポインタ演算がありません.
{\tacos}版では,
これらの代用となる関数が\|#include <crt0.hmm>|を書いた後で使用できます.
ここで紹介する関数はトランスレータ版では使用できません.

\subsection{{\ul}iToA関数}
\label{chap4:itoa}

整数から参照へ型を変換する関数です.
整数を引数に \|void[]| 参照(アドレス)を返します.
関数の値は \|void[]| 型の参照なので,
どのような参照型変数にも代入できます.

\begin{quote}
\begin{verbatim}
#include <crt0.hmm>
public void[] _iToA(int a);
\end{verbatim}
\end{quote}

\subsection{{\ul}aToI関数}

参照から整数へ型を変換する関数です.
参照(アドレス)を引数に整数を返します.
引数の型は \|void[]| なので,
参照型ならどんな型でも渡すことができます.

\begin{quote}
\begin{verbatim}
#include <crt0.hmm>
public int _aToI(void[] a);
\end{verbatim}
\end{quote}

\subsection{{\ul}aToA関数}

参照から参照へ型を変換する関数です.
異なる型の参照の間で代入をするために使用できます.

\begin{quote}
\begin{verbatim}
#include <crt0.hmm>
public void[] _aToA(void[] a);
\end{verbatim}
\end{quote}

\subsection{{\ul}addrAdd関数}

\cl のポインタ演算の代用にする関数です.
参照(アドレス)と整数を引数に渡し,
参照から整数バイト先の参照(アドレス)を返します.

\begin{quote}
\begin{verbatim}
#include <crt0.hmm>
public void[] _addrAdd(void[] a, int n);
\end{verbatim}
\end{quote}

\subsection{{\ul}aCmp関数}

参照(アドレス)の大小比較を行う関数です.
\cmml では参照の大小比較はできません.
\javal でも参照の大小比較はできないので,通常はこの仕様で十分と考えられます.
しかし,\|malloc|,\|free| 関数等の実現には
アドレスの大小比較が必要です.
そこで,アドレスの大小比較をする {\ul}aCmp 関数を用意しました.
{\ul}aCmp 関数は,\|a| の方が大きい場合は 1 を,
\|b| の方が大きい場合は -1 を,
\|a| と \|b| が等しい場合は 0 を返します.

\begin{quote}
\begin{verbatim}
#include <crt0.hmm>
public int _aCmp(void[] a, void[] b);
\end{verbatim}
\end{quote}

\subsection{{\ul}uCmp関数}

符号無し数の比較を行う関数です.
{\ul}uCmp 関数は,\|a| の方が大きい場合は 1 を,
\|b| の方が大きい場合は -1 を,
\|a| と \|b| が等しい場合は 0 を返します.

\begin{quote}
\begin{verbatim}
#include <crt0.hmm>
public int _uCmp(int a, int b);
\end{verbatim}
\end{quote}

\subsection{{\ul}args関数}
\label{chap4:args}

printf 関数のような可変個引数の関数を実現するために,
可変個引数関数の内部で引数を配列としてアクセスできるようにする関数です.
{\ul}args 関数は{\ul}args を呼び出した
{\cmm}関数の第1引数を添字\|0|とする\|int|配列を返します.

\begin{quote}
\begin{verbatim}
#include <crt0.hmm>
public int[] _args();
\end{verbatim}
\end{quote}

次に可変個引数関数の使用例を示します.

\begin{mylist}
\begin{verbatim}
int f(char[] s, ...) {         // ... は可変個引数の関数を表す
  int[] args = _args();        // args配列は引数配列を格納
  printf("%s\n", args[0]);     // 引数 s のこと(第1引数)
  printf("%d\n", args[1]);     // 引数 ... の最初に該当(第2引数)
  printf("%d\n", args[2]);     // 引数 ... の2番に該当(第3引数)
\end{verbatim}
\end{mylist}

\subsection{{\ul}add32関数}
\label{chap4:add32}

\tac で32ビットの計算を行うための関数です.
計算の対象は\|int|配列で上位\|[0]|,下位\|[1]|の順に表現した
符号なし32ビットデータです.
引数の\|dst|,\|src|が32ビットデータを表現する\|int|配列です.
$dst = dst + src$を計算します.
{\ul}add32 関数が返す値は\|dst|配列の参照です.

\begin{quote}
\begin{verbatim}
#include <crt0.hmm>
public int[] _add32(int[] dst, int[] src);
\end{verbatim}
\end{quote}

次に使用例を示します.
\|a|,\|b|配列が32ビットデータを表します.
この例は{\ul}add32 関数が\|dst|の参照を返すことを利用しています.

\begin{mylist}
\begin{verbatim}
int[] a = {12345,6789}:
int[] b = {23456,7890}:
...
  _add32(_add32(a, b), b);  // a = a + b + b;
\end{verbatim}
\end{mylist}

\subsection{{\ul}sub32関数}
\label{chap4:sub32}

\tac で32ビットの計算を行うための関数です.
$dst = dst - src$を計算します.
{\ul}sub32 関数が返す値は\|dst|配列の参照です.

\begin{quote}
\begin{verbatim}
#include <crt0.hmm>
public int[] _sub32(int[] dst, int[] src);
\end{verbatim}
\end{quote}

\subsection{{\ul}div32関数}
\label{chap4:div32}

\tac で32ビットの計算を行うための関数です.
$dst = dst / src$を計算します.
{\ul}div32 関数が返す値は\|dst|配列の参照です.

\begin{quote}
\begin{verbatim}
#include <crt0.hmm>
public int[] _div32(int[] dst, int src);
\end{verbatim}
\end{quote}

\subsection{{\ul}mod32関数}
\label{chap4:mod32}

\tac で32ビットの計算を行うための関数です.
\|dst|を\|src|で割った余りを返します.
上記の3つの関数と異なり関数の返り値と\|src|が\|int|型,
\|dst|の値は変化しないことに気を付けて下さい.

\begin{quote}
\begin{verbatim}
#include <crt0.hmm>
public int _mod32(int[] dst, int src);
\end{verbatim}
\end{quote}

\subsection{{\ul}in関数}
\label{chap4:in}

\tac のI/Oポートをアクセスする関数です.
I/O特権モードのアプリケーションプログラムだけが使用できます.
I/O空間の\|p|番地からワード(16ビット)のデータを入力します.

\begin{quote}
\begin{verbatim}
#include <crt0.hmm>
public int _in(int p);
\end{verbatim}
\end{quote}

\subsection{{\ul}out関数}
\label{chap4:out}

\tac のI/Oポートをアクセスする関数です.
I/O特権モードのアプリケーションプログラムだけが使用できます.
I/O空間の\|p|番地に\|v|のワード(16ビット)データを出力します.

\begin{quote}
\begin{verbatim}
#include <crt0.hmm>
public vod _out(int p, int v);
\end{verbatim}
\end{quote}
 % ライブラリ関数
%
%  5章 システムコール
%
\chapter{システムコール}

\tacos のシステムコールを呼び出す関数です.
{\bf トランスレータ版では使用できません.}
\|#include <syslib.hmm>|と書いた後で使用します.

\section{プロセス関連}

\tacos では,
\|exec|で新しいプロセスを作ると同時に新しいプログラムを実行します.
UNIXの\|fork-exec|方式とは異なります.

%\tac にはベースレジスタや多重仮想記憶のような機構がないので,
%\|fork|システムコールが実現できませんでした.

\subsection{exec}

\|path|でプログラムファイルを指定し,
新しいプロセスで新しいプログラムの実行を開始します.
\|argv|は,開始するプログラムの\|main|関数の
第2引数(\|char[][]argv|)に渡される文字列配列です.
\|envp|は,開始するプログラムに渡す環境変数です.
\|argv|と同じ要領で文字列配列を渡します.
\|exec|は正常なら\|0|,エラー発生なら負のエラー番号を返します.

\begin{quote}
\begin{verbatim}
#include <syslib.hmm>
public int exec(char[] path, void[] argv, void[] envp);
\end{verbatim}
\end{quote}

下に使用例を示します.
\|argv[0]|にプログラム名,
\|argv[1]|に第1引数のように格納します.
最後に\|null|を格納します.
環境変数に興味がない場合は,
自身の環境変数(\|environ|)を子プロセスに渡します.

\begin{mylist}
\begin{verbatim}
  #include <stdlib.hmm>
  #include <syslib.hmm>
  char[][] args = {"prog", "param1", "param2", null};
  public int main() {
    exec("/bin/prog.exe", args, environ);
    return 1;
  }
\end{verbatim}
\end{mylist}

子プロセス側のプログラム(prog.cmm)は次のようになります.

\begin{mylist}
\begin{verbatim}
public int main(int argc, char[][]argv, char[][]envp) {
  int c = argc;       // 前のプログラムで起動されたとき 3
  char[] s = argv[1]; // 前のプログラムで起動されたとき "param1"
  return 0;
}
\end{verbatim}
\end{mylist}

\subsection{\ul exit}

\|_exit|はプログラム(プロセス)を終了します.
\|_exit|は入出力のバッファをフラッシュしません.
\|_exit|は緊急終了用に使用し,
普通は標準ライブラリの\|exit|を使用します.

\|status|は,親プロセスに返す終了コードです.
\|0|が正常終了の意味,\|1|以上はユーザが決めた終了コード,
負の値は\tabref{chap4:err}に示す記号名で定義されています.
負の値を返すと親プロセスがシェルの場合,
シェル側でエラーメッセージを表示してくれます.

\begin{quote}
\begin{verbatim}
#include <syslib.hmm>
public void _exit(int status);
\end{verbatim}
\end{quote}

\subsection{wait}

\|wait|は子プロセスの終了を待ちます.
\|stat|には大きさ1の\|int|配列を渡します.
子プロセスが終了した際,\|stat[0]|に終了コードが書き込まれます.
\|wait|は正常なら\|0|,エラー発生なら負のエラー番号を返します.

\begin{quote}
\begin{verbatim}
#include <syslib.hmm>
public int wait(int[] stat);
\end{verbatim}
\end{quote}

\subsection{sleep}

\|sleep|はプロセスを指定された時間だけ停止します.
\|ms|はミリ秒単位で停止時間を指定します.
\|ms|に負の値を指定すると\|EINVAL|エラーになります.
それ以外では,\|sleep|は\|0|を返します.

\begin{quote}
\begin{verbatim}
#include <syslib.hmm>
public int sleep(int ms);
\end{verbatim}
\end{quote}

\section{ファイル操作}

\tacos は,マイクロSDカードのFAT16ファイルシステムを扱うことができます.
VFATには対応していません.

\subsection{creat}

\|creat|は新規ファイルを作成します.
\|path|は新しいファイルのパスです.
\|creat|は正常なら\|0|,エラー発生なら負のエラー番号を返します.

\begin{quote}
\begin{verbatim}
#include <syslib.hmm>
public int creat(char[] path);
\end{verbatim}
\end{quote}

\subsection{remove}

\|remove|はファイルを削除します.
\|path|は削除するファイルのパスです.
\|remove|は正常なら\|0|,エラー発生なら負のエラー番号を返します.

\begin{quote}
\begin{verbatim}
#include <syslib.hmm>
public int remove(char[] path);
\end{verbatim}
\end{quote}

\subsection{mkDir}

\|mkDir|は新規のディレクトリを作成します.
\|path|は新しいディレクトリのパスです.
\|mkDir|は正常なら\|0|,エラー発生なら負のエラー番号を返します.

\begin{quote}
\begin{verbatim}
#include <syslib.hmm>
public int mkDir(char[] path);
\end{verbatim}
\end{quote}

\subsection{rmDir}

\|rmDir|はディレクトリを削除します.
\|path|は削除するディレクトリのパスです.
\|rmDir|は正常なら\|0|,エラー発生なら負のエラー番号を返します.
削除するディレクトリが空でない場合はエラーになります.

\begin{quote}
\begin{verbatim}
#include <syslib.hmm>
public int rmDir(char[] path);
\end{verbatim}
\end{quote}

\subsection{stat}

\|stat|はファイルのメタデータを\|Stat|構造体に読み出します.
エラー発生なら負のエラー番号を返します.

\begin{quote}
\begin{verbatim}
#include <sys/fs.hmm>
#include <syslib.hmm>
public int stat(char[] path, Stat stat);
\end{verbatim}
\end{quote}

\|Stat|構造体は\|sys/fs.hmm|ファイル中で以下のように定義されます.
\|attr|はFAT16ファイルシステムの
ディレクトリエントリから読みだしたファイルの属性です.
詳しい意味はFAT16ファイルシステムの文献を参照してください.

\begin{quote}
\begin{verbatim}
struct Stat {   // FAT16ファイルシステムからファイルの情報を取り出す.
  int attr;     // read-only(0x01),hidden(0x02),directory(0x10) 他
  int clst;     // ファイルの開始クラスタ番号
  int lenH;     // ファイル長上位16ビット
  int lenL;     // ファイル長下位16ビット
};
\end{verbatim}
\end{quote}

\section{ファイルの読み書き}

ファイルの読み書きには,
通常は\pageref{chap4:stdio}ページの標準入出力ライブラリ関数を用います.
以下のシステムコールは,主にライブラリ関数の内部で使用されます.

\subsection{open}

\|open|はファイルを開きます
\|path|は開くファイルのパスです.
\|mode|には\|O_RDONLY|,\|O_WRONLY|,\|O_APPEND|のいずれかを指定します.
\|open|は正常なら非負のファイル記述子,
エラー発生なら負のエラー番号を返します.
ファイルが存在しない場合は,どのモードでもエラーになります.
新規ファイルに書き込みたい場合は,
事前に\|creat|システムコールを用いてファイルを作成しておく必要があります.

\|open|はディレクトリを\|O_RDONLY|モードで開くことができます.
ディレクトリは\pageref{chap4:readDir}ページの\|readDir|関数で読みます.

\begin{quote}
\begin{verbatim}
#include <syslib.hmm>
public int open(char[] path, int mode);
\end{verbatim}
\end{quote}

\subsection{close}

\|close|は\|open|で開いたファイルを閉じます.
\|fd|は閉じるファイルのファイル記述子です.
\|close|は正常なら\|0|,エラー発生なら負のエラー番号を返します.

\begin{quote}
\begin{verbatim}
#include <syslib.hmm>
public int close(int fd);
\end{verbatim}
\end{quote}

\subsection{read}

\|read|は\|open|を用い\|O_RDONLY|モードで開いたファイルから
データを読みます.
\|fd|はファイル記述子です.
\|buf|はデータを読み込むバッファ,
\|len|はバッファサイズ(バイト単位)です.
\|read|は正常なら読み込んだバイト数,
エラー発生なら負のエラー番号を返します.
EOFでは\|0|を返します.

\begin{quote}
\begin{verbatim}
#include <syslib.hmm>
public int read(int fd, void[] buf, int len);
\end{verbatim}
\end{quote}

\subsection{write}

\|write|は\|open|を用い\|O_WRONLY|,
\|O_APPEND|モードで開いたファイルへデータを書き込みます.
\|fd|はファイル記述子です.
\|buf|は書き込むデータが置いてあるバッファ,
\|len|は書き込むデータのサイズ(バイト単位)です.
\|write|は正常なら書き込んだバイト数,
エラー発生なら負のエラー番号を返します.

\begin{quote}
\begin{verbatim}
#include <syslib.hmm>
public int write(int fd, void[] buf, int len);
\end{verbatim}
\end{quote}

\subsection{seek}

\|seek|は\|open|を用い開いたファイルの読み書き位置を変更します.
\|fd|はファイル記述子です.
seek位置は,上位16bit(\|ptrh|)と
下位16bit(\|ptrl|)を組み合わせて指定します.
\|seek|は正常なら\|0|,
エラー発生なら負のエラー番号を返します.

\begin{quote}
\begin{verbatim}
#include <syslib.hmm>
public int seek(int fd, int ptrh, int ptrl);
\end{verbatim}
\end{quote}

\section{コンソール関連}

コンソール入出力には,
通常は\pageref{chap4:stdio}ページの標準入出力ライブラリ関数を用います.
以下のシステムコールは,主にライブラリ関数の内部で使用されます.
\|ttyRead|はライブラリ関数が\|stdin|からの読み込みをする場合に,
\|ttyWrite|はライブラリ関数が\|stdout|,\|stderr|への
書き込みをする場合にライブラリ関数内部で使用されます.

\subsection{ttyRead}

\|ttyRead|はキーボードから1行入力します.
読み込んだ内容は\|buf|で指定されるバッファに格納されます.
\|len|は\|buf|のバイト数です.
読み込んだ内容の最後に\|'\0'|は含まれませんが\|'\n'|は含まれます.
\|ttyRead|はキーボードから入力した文字数を返します.

\begin{quote}
\begin{verbatim}
#include <syslib.hmm>
public int ttyRead(void[] buf, int len);
\end{verbatim}
\end{quote}

\subsection{ttyWrite}

\|ttyWrite|は\|buf|の内容を画面に出力します.

\begin{quote}
\begin{verbatim}
#include <syslib.hmm>
public int ttyWrite(void[] buf, int len);
\end{verbatim}
\end{quote}

\subsection{ttyCtl}

\|ttyClt|はコンソールのモードを操作します.

\begin{quote}
\begin{verbatim}
#include <syslib.hmm>
public int ttyCtl(int cmd, int mode);
\end{verbatim}
\end{quote}

\|cmd|には\|TTYCTL_GETMODE|,\|TTYCTL_SETMODE|のどちらかを指定します.
\|TTYCTL_GETMODE|を指定した場合,\|ttyCtl|は現在のモードを返します.
\|TTYCTL_SETMODE|を指定した場合,\|ttyCtl|はモードを\|mode|に変更します.
\|ttyCtl|の返り値と\|mode|では,
モードを\|TTYCTL_MODE_COOKED|,\|TTYCTL_MODE_ECHO|,\|TTYCTL_MODE_NBLOCK|の
ビットマップで表現します.

\|TTYCTL_MODE_COOKED|は通常ONになっています.
このモードがONになっている場合,
コンソール入出力で\|'\r'|と\|'\n'|の間で適切な変換が行われます.
また,\|'\b'|(バックスペースキー)を用いた行編集ができます.

\|TTYCTL_MODE_ECHO|も通常ONになっています.
このモードがONになっている場合,
キーボードから入力した文字が画面にエコーバックされます.

\|TTYCTL_MODE_NBLOCK|は通常OFFになっています.
このモードがONになっている場合,\|ttyRead|が入力待ちになりません.

以下にプログラム例を示します.

\begin{quote}
\begin{verbatim}
int mode = ttyCtl(TTYCTL_GETMODE, 0);      // 現在のモードを取得
int noechoMode = mode & ~TTYCTL_MODE_ECHO;
ttyCtl(TTYCTL_SETMODE, noechoMode);       // NOECHOモードに変更
...
ttyCtl(TTYCTL_SETMODE, mode);             // 最初の状態に戻す
\end{verbatim}
\end{quote}
 % システムコール

\appendix
\include{syntax}  % 付録 文法まとめ
% 
% 付録:コマンドリファレンス
%
\chapter{コマンドリファレンス}
\label{app:command}

%UNIX や macOS 上で動作する{\cmm}コンパイラの使用方法を解説します.

\section{{\cme}コマンド}

{\cmm}プログラムを{\tac}で実行できる{\tt .exe}ファイルに変換します.
{\cme}コマンドは,内部で「\ref{command:cmmc} {\cmmc}コマンド」や
「{\tt Util--}ユーティリティ」プログラムを自動的に呼び出すシェルスクリプトです.

\begin{flushleft}
\begin{tabular}{l l}
{\bf 形式 : } & \|cm2e [-h] [-o exec] [-S] [-c] [-E] [-K] [-P] | \\
              & ~~~ \| [-nostdinc] [-I <dir>] [-c] [-Dxx=yy] <file>...|
\end{tabular}
\end{flushleft}

{\tt <file>...}の各ファイルについて,
プリプロセッサ({\tt cpp}),
コンパイラ({\tt c--}),
アセンブラ({\tt as--})を順に呼び出し,
リロケータブルオブジェクト(「{\tt Util--}解説書」参照)に変換します.
次に,
リンカー({\tt ld--})を用いリロケータブルオブジェクトを結合します.
最後に,実行可能形式作成プログラム({\tt objexe--})を呼び出し
{\tt .exe}ファイルを作成します.

{\cme}は,
指定されたファイルの拡張子からファイルの種類を判断し,
必要な処理を自動的に実行します.
拡張子「{\tt .cmm}」は{\cmml}のソースプログラム,
「{\tt .s}」は{\tac}のアセンブリ言語プログラム,
「{\tt .o}」は{\tac}のリロケータブルオブジェクトと判断します.

\begin{quote}
\hspace{-1em}以下のオプションが使用できます.

\begin{description}
\item[{\tt -h}] : 使用方法メッセージを表示します.
\item[{\tt -o}] : 作成する{\tt .exe}ファイルの名前を指定します.
{\tt -o}オプションの後ろに空白で区切ってファイル名を入力します.
\item[{\tt -S}] : アセンブラソースプログラム{\tt .s}の作成まで行い,
  それより後の処理を行いません.
\item[{\tt -c}] : リロケータブルオブジェクトファイル{\tt .o}の作成まで行い,
  それより後の処理を行いません.
\item[{\tt -E}] : プリプロセッサで処理した{.cmm}ファイルの内容を
  標準出力ストリームに書き出します.
\item[{\tt -K}] : {\tacos}カーネル用モードでコンパイルを行います.
  コンパイル結果に,
  ユーザプログラム用のスタックオーバーフローチェック機能を埋め込みません.
\item[{\tt -P}] : I/O特権モードの{\tt .exe}ファイルを作ります.
  {\tacos}のI/O特権モードユーザプログラムは,
  \|IN|,\|OUT|機械語命令を実行することができます.
\item[{\tt -nostdinc}] : 標準のインクルードディレクトリを使用しません.
\item[{\tt -I}] : インクルードディレクトリを追加します.
このオプションを繰り返し使用することで複数のディレクトリを追加できます.
\item[{\tt -D}] : このオプションは,そのままプリプロセッサに渡されます.
次に使用例を示します.\\
\|$ cm2e -DDEBUG=1 -o hello hello.cmm|
\end{description}
\end{quote}

\section{{\cmc}コマンド}

{\cmm}プログラムを{\cl}プログラムに変換した後,
{\cl}プログラムをコンパイルしてUNIXやmacOSの実行形式を作成します.
{\cmc}コマンドは,必要なプログラムを自動的に呼び出すシェルスクリプトです.

\begin{flushleft}
\begin{tabular}{l l}
{\bf 形式 : } & \|cm2c [-h] [-o exec] [-S] [-c] [-E] [-nostdinc] [-rtc] | \\
              & ~~~ \| [-nortc] [-I <dir>] [-Dxx=yy] <file>...|
\end{tabular}
\end{flushleft}

{\tt <file>...}の各ファイルについて,
プリプロセッサ({\tt cpp}),
トランスレータ({\tt c-c--}または{\tt rtc-c--}),
{\tt C}コンパイラドライバ({\tt cc})を順に呼び出し,
UNIXやmacOSのリロケータブルオブジェクトに変換します.
次に,{\tt C}コンパイラドライバ({\tt cc})を呼び出し,
リロケータブルオブジェクトを結合し実行可能ファイルを作成します.

{\cmc}は,
指定されたファイルの拡張子からファイルの種類を判断し,
必要な処理を自動的に実行します.
拡張子「{\tt .cmm}」は{\cmml}のソースプログラム,
「{\tt .c}」は{\cl}ソースプログラム,
「{\tt .o}」はUNIXやmacOSのリロケータブルオブジェクトと判断します.

\begin{quote}
\hspace{-1em}以下のオプションが使用できます.

\begin{description}
\item[{\tt -h}] : 使用方法メッセージを表示します.
\item[{\tt -o}] : 作成する実行可能ファイルの名前を指定します.
{\tt -o}オプションの後ろに空白で区切ってファイル名を入力します.
\item[{\tt -S}] : {\cl}ソースプログラム{\tt .c}の作成まで行い,
  それより後の処理を行いません.
\item[{\tt -c}] : リロケータブルオブジェクトファイル{\tt .o}の作成まで行い,
  それより後の処理を行いません.
\item[{\tt -E}] : プリプロセッサで処理した{.cmm}ファイルの内容を
  標準出力ストリームに書き出します.
\item[{\tt -nostdinc}] : 標準のインクルードディレクトリを使用しません.
\item[{\tt -rtc}] : 実行時エラーチェックを行う実行形式を作成します.
  実行時エラーチェックの内容は,\|null|参照の使用と配列の添字範囲チェックです.
  デフォルトが{\tt -rtc}です.
\item[{\tt -nortc}] : 実行時エラーチェックを行わない実行形式を作成します.
\item[{\tt -I}] : インクルードディレクトリを追加します.
  このオプションを繰り返し使用することで複数のディレクトリを追加できます.
\item[{\tt -D}] : このオプションは,そのままプリプロセッサに渡されます.
\end{description}
\end{quote}

\section{{\cmi}コマンド}

{\cmm}プログラムをコンパイルして中間言語に変換します.
{\cmi}コマンドは,
内部で「\ref{command:vcmmc} {\icmmc}コマンド」を呼び出すシェルスクリプトです.

\begin{flushleft}
{\bf 形式 : }~~~\|cm2i [-h] [-E] [-nostdinc] [-I <dir>] [-Dxx=yy] <file>...|
\end{flushleft}

{\tt <file>...}の各ファイルについて,
プリプロセッサ({\tt cpp}),
コンパイラ({\tt ic-c--})
を順に呼び出し中間言語({\tt .i})を出力します.
{\cmi}コマンドに指定できるファイルは,
拡張子「{\tt .cmm}」の{\cmml}ソースプログラムだけです.

\begin{quote}
\hspace{-1em}以下のオプションが使用できます.

\begin{description}
\item[{\tt -h}] : 使用方法メッセージを表示します.
\item[{\tt -E}] : プリプロセッサで処理した{.cmm}ファイルの内容を
標準出力ストリームに書き出します.
それより後の処理を行いません.
\item[{\tt -nostdinc}] : 標準のインクルードディレクトリを使用しません.
\item[{\tt -I}] : インクルードディレクトリを追加します.
このオプションを繰り返し使用することで複数のディレクトリを追加できます.
\item[{\tt -D}] : このオプションは,そのままプリプロセッサに渡されます.
\end{description}
\end{quote}

\section{{\cmv}コマンド}

{\cmm}プログラムをコンパイルして仮想スタックマシンのニーモニックに変換します.
{\cmv}コマンドは,
内部で「\ref{command:vcmmc} {\vcmmc}コマンド」を呼び出すシェルスクリプトです.

\begin{flushleft}
{\bf 形式 : }~~~\|cm2v [-h] [-E] [-nostdinc] [-I <dir>] [-Dxx=yy] <file>...|
\end{flushleft}

{\tt <file>...}の各ファイルについて,
プリプロセッサ({\tt cpp}),
コンパイラ({\tt vm-c--})
を順に呼び出し仮想スタックマシンのニーモニック({\tt .v})を出力します.
{\cmv}コマンドに指定できるファイルは,
拡張子「{\tt .cmm}」の{\cmml}ソースプログラムだけです.

\begin{quote}
\hspace{-1em}以下のオプションが使用できます.

\begin{description}
\item[{\tt -h}] : 使用方法メッセージを表示します.
\item[{\tt -E}] : プリプロセッサで処理した{.cmm}ファイルの内容を
標準出力ストリームに書き出します.
それより後の処理を行いません.
\item[{\tt -nostdinc}] : 標準のインクルードディレクトリを使用しません.
\item[{\tt -I}] : インクルードディレクトリを追加します.
このオプションを繰り返し使用することで複数のディレクトリを追加できます.
\item[{\tt -D}] : このオプションは,そのままプリプロセッサに渡されます.
\end{description}
\end{quote}

\section{{\cmmc}コマンド}
\label{command:cmmc}

{\cmml}の{\tac}用コンパイラです.
{\bf 通常は{\cme}から起動されユーザが直接使用することはありません.}
\cmml で記述されたプログラムを入力し,
\tac アセンブリ言語で記述したプログラムに変換します.
\cmmc コマンドの書式は次の通りです.

\begin{flushleft}
{\bf 形式 : }~~~\|c-- [-h] [-v] [-O0] [-O] [-O1] [-K] [<source file>]|\\
({\bf 注意}:オプションは書式の順番で指定する必要があります.)
\end{flushleft}

引数に \cmml のソースプログラムファイルを指定した場合は,
指定されたファイルからソースプログラムを読み込みます.
ファイルが省略された場合は標準入力ストリームからソースプログラムを読み込みます.
どちらの場合もコンパイル結果は標準出力ストリームに出力します.
ソースプログラムファイルの拡張子は「\|.cmm|」にします.

\|-h|,\|-v|オプションは使用方法メッセージを表示します.
\|-O0|オプションを指定すると,
ソースコード中の定数式をコンパイル時に計算したり,
実行されることがないプログラムの部分を削除したりする等の最適化をしません.
\|-O|,\|-O1|は最適化を促すオプションですが,
デフォルトで\|ON|になっているので指定する必要はありません.
\|-K|オプションを使うと,
関数入口へのスタックオーバーフロー検出コードの埋め込みが抑制されます.
\tacos のカーネルをコンパイルするときに使用するオプションです.

\section{{\ccmmc}コマンド}
\label{command:ccmmc}

{\cmm}プログラムを{\cl}に変換して出力するトランスレータです.
{\bf 通常は{\cmc}から起動されユーザが直接使用することはありません.}

\begin{flushleft}
{\bf 形式 : }~~~\|c-c-- [-h] [-v] [-O0] [-O] [-O1] [-K] [<source file>]|\\
({\bf 注意}:オプションは書式の順番で指定する必要があります.)
\end{flushleft}

引数の意味は{\cmmc}コマンドと同様です.

\section{{\rtccmmc}コマンド}
\label{command:rtccmmc}

{\cmm}プログラムを{\cl}に変換して出力するトランスレータです.
{\bf 通常は{\cmc}から起動されユーザが直接使用することはありません.}
{\rtccmmc}は{\ccmmc}と異なり,
ユーザプログラムが\|null|参照を使用したり,
範囲外の添字を用いて配列をアクセスしていないかチェックする
実行時エラーチェック用のコードを出力に埋め込みます.

\begin{flushleft}
{\bf 形式 : }~~~\|rtc-c-- [-h] [-v] [-O0] [-O] [-O1] [-K] [<source file>]|\\
({\bf 注意}:オプションは書式の順番で指定する必要があります.)
\end{flushleft}

引数の意味は{\cmmc}コマンドと同様です.

\section{{\icmmc}コマンド}
\label{command:icmmc}

中間言語を出力する{\cmm}コンパイラです.
{\bf 通常は{\cmi}から起動されユーザが直接使用することはありません.}

中間言語の仕様は,\pageref{app:vm}ページに掲載してあります.
コンパイラの仕組みを学習したいときに利用します.
{\icmmc}コマンドの書式は次の通りです.

\begin{flushleft}
{\bf 形式 : }~~~\|ic-c-- [-h] [-v] [-O0] [-O] [-O1] [-K] [<source file>]|
\end{flushleft}

引数の意味は{\cmmc}コマンドと同様です.

\section{{\vcmmc}コマンド}
\label{command:vcmmc}

仮想スタックマシンのニーモニックを出力する{\cmm}コンパイラです.
{\bf 通常は{\cmv}から起動されユーザが直接使用することはありません.}

仮想スタックマシンのニーモニックは,
コンパイラ内部で用いている中間言語(\pageref{app:vm}ページ参照)と,
ほぼ一対一に対応します.
中間言語や仮想スタックマシンを学習したいときに利用します.
{\vcmmc}コマンドの書式は次の通りです.

\begin{flushleft}
{\bf 形式 : }~~~\|vm-c-- [-h] [-v] [-O0] [-O] [-O1] [-K] [<source file>]|
\end{flushleft}

引数の意味は{\cmmc}コマンドと同様です.
%次の実行例は変換結果を画面に表示しています.
%
%\begin{mylist}
%\begin{verbatim}
%$ cc -E -std=c99 -nostdinc -I/usr/local/cmmInclude \
% - < hello.cmm | vm-c--
%_stdin
%        WS      1
%_stdout
%        WS      1
%_stderr
%        WS      1
%.L1
%        STRING  "hello,world\n"
%_main
%        ENTRY   0
%        LDC     .L1
%        ARG
%        CALLF   1,_printf
%        POP
%        LDC     0
%        MREG
%        RET
%\end{verbatim}
%\end{mylist}

 % 付録 コマンドリファレンス
\include{vm}      % 付録 中間言語

% 発行元など
\backmatter
\pagestyle{empty}
\onecolumn
~
\subsubsection{変更履歴}
\begin{flushleft}
2023年08月20日 v.4.4.0  char → int の自動型変換を導入 \\
2021年03月18日 v.3.3.0  トランスレータ版の実行時エラーチェックを正式導入 \\
2020年12月11日 v.3.2.9  \|TTYCTL_MODE_NBLOCK|追記 \\
2020年06月30日 v.3.2.3  カレントディレクトリ追記 \\
2019年12月10日 v.3.2.2  環境変数ライブラリ追記,Statシステムコール追記 \\
2019年10月18日 v.3.2.1  conRead/WriteをttyRead/Writeに変更,ttyCtlを追記 \\
2019年03月24日 v.3.2.0  トランスレータ版に実行時エラーチェックを試験的に導入 \\
2019年01月27日 v.3.1.12 メモリ保護違反,I/O特権モード,in(),out(),htoi()追記 \\
2018年11月18日 V.3.1.11 fsize、fseek、printf("\%ld")、32ビット演算を追記 \\
2018年01月27日 V.3.1.10 ドライバの -nostdinc -I オプションを追記 \\
2016年10月09日 V.3.1.7 cm2iの記述を追加、文法まとめを改良 \\
2016年09月18日 V.3.1.6 中間言語の仕様を変更 \\
2016年09月10日 V.3.1.2a 文字列クラス分類関数にトランスレータの場合を追記 \\
2016年08月13日 \cmm V.3.1.2 用(feofの仕様変更)  \\
2016年06月03日 \cmm V.3.1.0 用作成  \\
2016年03月15日 \cmm V.3.0.0 用作成  \\
\end{flushleft}

\vfil

\subsubsection{対応ソフトウェアのバージョン}
\begin{tabular}{|l|l|}
\hline
{\tt C--}      & Ver.4.0.0 \\
\hline
{\tt TacOS}    & Ver.4.0.2 \\
\hline
\end{tabular}

\vfill\vfill
\begin{center}
\fbox{\parbox{10cm}{ \vspace{0.3cm}
\large{\bf プログラミング言語 \cmm} \\
\\
 発行年月 2023年8月 \ver \\
 著  者 重村 哲至 \\
 発  行 独立行政法人国立高等専門学校機構 \\
      徳山工業高等専門学校 \\
      情報電子工学科 \\
      〒745-8585 山口県周南市学園台 \\
      sigemura@tokuyama.ac.jp \\
}}
\end{center}
\vfill
\end{document}
